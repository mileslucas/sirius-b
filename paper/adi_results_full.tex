
\begin{figure}[h!]
    \centering
    \epsscale{0.9}
    \plotone{2020feb04_contrast_curves}
    \caption{5$\sigma$ contrast curves from every ADI algorithm for the first epoch. Both the Gaussian (solid lines) and Student-t corrected (dashed lines) contrast curves are shown. }
\end{figure}

\begin{figure}[h!]
    \centering
    \epsscale{0.9}
    \plotone{2020nov21_contrast_curves}
    \caption{5$\sigma$ contrast curves from every ADI algorithm for the second epoch. Both the Gaussian (solid lines) and Student-t corrected (dashed lines) contrast curves are shown. }
\end{figure}

\begin{figure}[h!]
    \centering
    \epsscale{0.9}
    \plotone{2020nov28_contrast_curves}
    \caption{5$\sigma$ contrast curves from every ADI algorithm for the third epoch. Both the Gaussian (solid lines) and Student-t corrected (dashed lines) contrast curves are shown. }
\end{figure}

%%%%%%%%%%%%%%%%%%%%%%%%%%%%%%%%%%%%%%%%%%%%%%%%

\figsetstart
\figsetnum{10}
\figsettitle{ADI processing results}

%%% first epoch

\figsetgrpstart
\figsetgrpnum{10.1}
\figsetgrptitle{2020-02-04 median}
\figsetplot{reports/2020feb04_median}
\figsetgrpnote{Post-processing results from the first epoch using full-frame median subtraction. The top-left frame is the collapsed residual frame, the top-right is the Gaussian S/N map, the bottom-left is the STIM probability map, and the bottom-right is the Student-t corrected significance map. In each frame the inner FWHM is masked out. The right figure show the Gaussian (solid line) and Student-t corrected (dashed curve) 5$\sigma$ contrast curve.}
\figsetgrpend

\figsetgrpstart
\figsetgrpnum{10.2}
\figsetgrptitle{2020-02-04 PCA(1)}
\figsetplot{reports/2020feb04_pca-1}
\figsetgrpnote{Post-processing results from the first epoch using full-frame PCA subtraction with 1 component. The top-left frame is the collapsed residual frame, the top-right is the Gaussian S/N map, the bottom-left is the STIM probability map, and the bottom-right is the Student-t corrected significance map. In each frame the inner FWHM is masked out. The right figure show the Gaussian (solid line) and Student-t corrected (dashed curve) 5$\sigma$ contrast curve.}
\figsetgrpend

\figsetgrpstart
\figsetgrpnum{10.3}
\figsetgrptitle{2020-02-04 NMF(1)}
\figsetplot{reports/2020feb04_nmf-1}
\figsetgrpnote{Post-processing results from the first epoch using full-frame NMF subtraction with 1 component. The top-left frame is the collapsed residual frame, the top-right is the Gaussian S/N map, the bottom-left is the STIM probability map, and the bottom-right is the Student-t corrected significance map. In each frame the inner FWHM is masked out. The right figure show the Gaussian (solid line) and Student-t corrected (dashed curve) 5$\sigma$ contrast curve.}
\figsetgrpend

\figsetgrpstart
\figsetgrpnum{10.4}
\figsetgrptitle{2020-02-04 GreeDS(1)}
\figsetplot{reports/2020feb04_greeds-1}
\figsetgrpnote{Post-processing results from the first epoch using full-frame GreeDS subtraction with 1 component. The top-left frame is the collapsed residual frame, the top-right is the Gaussian S/N map, the bottom-left is the STIM probability map, and the bottom-right is the Student-t corrected significance map. In each frame the inner FWHM is masked out. The right figure show the Gaussian (solid line) and Student-t corrected (dashed curve) 5$\sigma$ contrast curve.}
\figsetgrpend

\figsetgrpstart
\figsetgrpnum{10.5}
\figsetgrptitle{2020-02-04 annular median}
\figsetplot{reports/2020feb04_annular_median}
\figsetgrpnote{Post-processing results from the first epoch using annular median subtraction with a rotation threshold of 0.5 FWHM. The top-left frame is the collapsed residual frame, the top-right is the Gaussian S/N map, the bottom-left is the STIM probability map, and the bottom-right is the Student-t corrected significance map. In each frame the inner FWHM is masked out. The right figure show the Gaussian (solid line) and Student-t corrected (dashed curve) 5$\sigma$ contrast curve.}
\figsetgrpend

\figsetgrpstart
\figsetgrpnum{10.6}
\figsetgrptitle{2020-02-04 annular PCA(1)}
\figsetplot{reports/2020feb04_annular_pca-1}
\figsetgrpnote{Post-processing results from the first epoch using annular PCA subtraction with 1 component and a rotation threshold of 0.5 FWHM. The top-left frame is the collapsed residual frame, the top-right is the Gaussian S/N map, the bottom-left is the STIM probability map, and the bottom-right is the Student-t corrected significance map. In each frame the inner FWHM is masked out. The right figure show the Gaussian (solid line) and Student-t corrected (dashed curve) 5$\sigma$ contrast curve.}
\figsetgrpend

\figsetgrpstart
\figsetgrpnum{10.7}
\figsetgrptitle{2020-02-04 annular NMF(1)}
\figsetplot{reports/2020feb04_annular_nmf-1}
\figsetgrpnote{Post-processing results from the first epoch using annular NMF subtraction with 1 component and a rotation threshold of 0.5 FWHM. The top-left frame is the collapsed residual frame, the top-right is the Gaussian S/N map, the bottom-left is the STIM probability map, and the bottom-right is the Student-t corrected significance map. In each frame the inner FWHM is masked out. The right figure show the Gaussian (solid line) and Student-t corrected (dashed curve) 5$\sigma$ contrast curve.}
\figsetgrpend


%%% second epoch

\figsetgrpstart
\figsetgrpnum{10.8}
\figsetgrptitle{2020-11-21 median}
\figsetplot{reports/2020nov21_median}
\figsetgrpnote{Post-processing results from the second epoch using full-frame median subtraction. The top-left frame is the collapsed residual frame, the top-right is the Gaussian S/N map, the bottom-left is the STIM probability map, and the bottom-right is the Student-t corrected significance map. In each frame the inner FWHM is masked out. The right figure show the Gaussian (solid line) and Student-t corrected (dashed curve) 5$\sigma$ contrast curve.}
\figsetgrpend

\figsetgrpstart
\figsetgrpnum{10.9}
\figsetgrptitle{2020-11-21 PCA(3)}
\figsetplot{reports/2020nov21_pca-3}
\figsetgrpnote{Post-processing results from the second epoch using full-frame PCA subtraction with 3 components. The top-left frame is the collapsed residual frame, the top-right is the Gaussian S/N map, the bottom-left is the STIM probability map, and the bottom-right is the Student-t corrected significance map. In each frame the inner FWHM is masked out. The right figure show the Gaussian (solid line) and Student-t corrected (dashed curve) 5$\sigma$ contrast curve.}
\figsetgrpend

\figsetgrpstart
\figsetgrpnum{10.10}
\figsetgrptitle{2020-11-21 NMF(3)}
\figsetplot{reports/2020nov21_nmf-3}
\figsetgrpnote{Post-processing results from the second epoch using full-frame NMF subtraction with 3 components. The top-left frame is the collapsed residual frame, the top-right is the Gaussian S/N map, the bottom-left is the STIM probability map, and the bottom-right is the Student-t corrected significance map. In each frame the inner FWHM is masked out. The right figure show the Gaussian (solid line) and Student-t corrected (dashed curve) 5$\sigma$ contrast curve.}
\figsetgrpend

\figsetgrpstart
\figsetgrpnum{10.11}
\figsetgrptitle{2020-11-21 GreeDS(3)}
\figsetplot{reports/2020nov21_greeds-3}
\figsetgrpnote{Post-processing results from the second epoch using full-frame GreeDS subtraction with 3 components. The top-left frame is the collapsed residual frame, the top-right is the Gaussian S/N map, the bottom-left is the STIM probability map, and the bottom-right is the Student-t corrected significance map. In each frame the inner FWHM is masked out. The right figure show the Gaussian (solid line) and Student-t corrected (dashed curve) 5$\sigma$ contrast curve.}
\figsetgrpend

\figsetgrpstart
\figsetgrpnum{10.12}
\figsetgrptitle{2020-11-21 annular median}
\figsetplot{reports/2020nov21_annular_median}
\figsetgrpnote{Post-processing results from the second epoch using annular median subtraction with a rotation threshold of 0.5 FWHM. The top-left frame is the collapsed residual frame, the top-right is the Gaussian S/N map, the bottom-left is the STIM probability map, and the bottom-right is the Student-t corrected significance map. In each frame the inner FWHM is masked out. The right figure show the Gaussian (solid line) and Student-t corrected (dashed curve) 5$\sigma$ contrast curve.}
\figsetgrpend

\figsetgrpstart
\figsetgrpnum{10.13}
\figsetgrptitle{2020-11-21 annular PCA(3)}
\figsetplot{reports/2020nov21_annular_pca-3}
\figsetgrpnote{Post-processing results from the second epoch using annular PCA subtraction with 3 components and a rotation threshold of 0.5 FWHM. The top-left frame is the collapsed residual frame, the top-right is the Gaussian S/N map, the bottom-left is the STIM probability map, and the bottom-right is the Student-t corrected significance map. In each frame the inner FWHM is masked out. The right figure show the Gaussian (solid line) and Student-t corrected (dashed curve) 5$\sigma$ contrast curve.}
\figsetgrpend

\figsetgrpstart
\figsetgrpnum{10.14}
\figsetgrptitle{2020-11-21 annular NMF(3)}
\figsetplot{reports/2020nov21_annular_nmf-3}
\figsetgrpnote{Post-processing results from the second epoch using annular NMF subtraction with 3 components and a rotation threshold of 0.5 FWHM. The top-left frame is the collapsed residual frame, the top-right is the Gaussian S/N map, the bottom-left is the STIM probability map, and the bottom-right is the Student-t corrected significance map. In each frame the inner FWHM is masked out. The right figure show the Gaussian (solid line) and Student-t corrected (dashed curve) 5$\sigma$ contrast curve.}
\figsetgrpend

%%% third epoch

\figsetgrpstart
\figsetgrpnum{10.15}
\figsetgrptitle{2020-11-28 median}
\figsetplot{reports/2020nov28_median}
\figsetgrpnote{Post-processing results from the third epoch using full-frame median subtraction. The top-left frame is the collapsed residual frame, the top-right is the Gaussian S/N map, the bottom-left is the STIM probability map, and the bottom-right is the Student-t corrected significance map. In each frame the inner FWHM is masked out. The right figure show the Gaussian (solid line) and Student-t corrected (dashed curve) 5$\sigma$ contrast curve.}
\figsetgrpend

\figsetgrpstart
\figsetgrpnum{10.16}
\figsetgrptitle{2020-11-28 PCA(2)}
\figsetplot{reports/2020nov28_pca-2}
\figsetgrpnote{Post-processing results from the third epoch using full-frame PCA subtraction with 2 components. The top-left frame is the collapsed residual frame, the top-right is the Gaussian S/N map, the bottom-left is the STIM probability map, and the bottom-right is the Student-t corrected significance map. In each frame the inner FWHM is masked out. The right figure show the Gaussian (solid line) and Student-t corrected (dashed curve) 5$\sigma$ contrast curve.}
\figsetgrpend

\figsetgrpstart
\figsetgrpnum{10.17}
\figsetgrptitle{2020-11-28 NMF(2)}
\figsetplot{reports/2020nov28_nmf-2}
\figsetgrpnote{Post-processing results from the third epoch using full-frame NMF subtraction with 2 components. The top-left frame is the collapsed residual frame, the top-right is the Gaussian S/N map, the bottom-left is the STIM probability map, and the bottom-right is the Student-t corrected significance map. In each frame the inner FWHM is masked out. The right figure show the Gaussian (solid line) and Student-t corrected (dashed curve) 5$\sigma$ contrast curve.}
\figsetgrpend

\figsetgrpstart
\figsetgrpnum{10.18}
\figsetgrptitle{2020-11-28 GreeDS(2)}
\figsetplot{reports/2020nov28_greeds-2}
\figsetgrpnote{Post-processing results from the third epoch using full-frame GreeDS subtraction with 2 components. The top-left frame is the collapsed residual frame, the top-right is the Gaussian S/N map, the bottom-left is the STIM probability map, and the bottom-right is the Student-t corrected significance map. In each frame the inner FWHM is masked out. The right figure show the Gaussian (solid line) and Student-t corrected (dashed curve) 5$\sigma$ contrast curve.}
\figsetgrpend

\figsetgrpstart
\figsetgrpnum{10.19}
\figsetgrptitle{2020-11-28 annular median}
\figsetplot{reports/2020nov28_annular_median}
\figsetgrpnote{Post-processing results from the third epoch using annular median subtraction with a rotation threshold of 0.5 FWHM. The top-left frame is the collapsed residual frame, the top-right is the Gaussian S/N map, the bottom-left is the STIM probability map, and the bottom-right is the Student-t corrected significance map. In each frame the inner FWHM is masked out. The right figure show the Gaussian (solid line) and Student-t corrected (dashed curve) 5$\sigma$ contrast curve.}
\figsetgrpend

\figsetgrpstart
\figsetgrpnum{10.20}
\figsetgrptitle{2020-11-28 annular PCA(2)}
\figsetplot{reports/2020nov28_annular_pca-2}
\figsetgrpnote{Post-processing results from the third epoch using annular PCA subtraction with 2 components and a rotation threshold of 0.5 FWHM. The top-left frame is the collapsed residual frame, the top-right is the Gaussian S/N map, the bottom-left is the STIM probability map, and the bottom-right is the Student-t corrected significance map. In each frame the inner FWHM is masked out. The right figure show the Gaussian (solid line) and Student-t corrected (dashed curve) 5$\sigma$ contrast curve.}
\figsetgrpend

\figsetgrpstart
\figsetgrpnum{10.21}
\figsetgrptitle{2020-11-28 annular NMF(2)}
\figsetplot{reports/2020nov28_annular_nmf-2}
\figsetgrpnote{Post-processing results from the third epoch using annular NMF subtraction with 2 components and a rotation threshold of 0.5 FWHM. The top-left frame is the collapsed residual frame, the top-right is the Gaussian S/N map, the bottom-left is the STIM probability map, and the bottom-right is the Student-t corrected significance map. In each frame the inner FWHM is masked out. The right figure show the Gaussian (solid line) and Student-t corrected (dashed curve) 5$\sigma$ contrast curve.}
\figsetgrpend

\figsetend

%%%%%%%%%%%%%%%%%%%%%%%%%%%%%%%%%%%%%%%%%%%%%%%%

\begin{figure}[h!]
    \centering
    \figurenum{10}
    \epsscale{0.9}
    \plotone{reports/2020feb04_median}
    \caption{Post-processing results from the second epoch using full-frame median subtraction. The top-left frame is the collapsed residual frame, the top-right is the Gaussian S/N map, the bottom-left is the STIM probability map, and the bottom-right is the Student-t corrected significance map. In each frame the inner FWHM is masked out. The right figure show the Gaussian (solid line) and Student-t corrected (dashed curve) 5$\sigma$ contrast curve. Outputs for other epochs and other algorithms (21 figures) are in the online figure set and the GitHub repository.}
\end{figure}

%%%%%%%%%%%%%%%%%%%%%%%%%%%%%%%%%%%%%%%%%%%%%%%%

\figsetstart
\figsetnum{11}
\figsettitle{PCA, NMF, and GreeDS mosaics}

%%% first epoch

\figsetgrpstart
\figsetgrpnum{11.1}
\figsetgrptitle{2020-02-04 PCA mosaic}
\figsetplot{reports/2020feb04_pca_mosaic}
\figsetgrpnote{Collapsed residual frames from the first epoch using PCA reduction with 1-10 components. The figures share a common scale and the inner FWHM is masked out for all the frames.}
\figsetgrpend

\figsetgrpstart
\figsetgrpnum{11.2}
\figsetgrptitle{2020-02-04 NMF mosaic}
\figsetplot{reports/2020feb04_nmf_mosaic}
\figsetgrpnote{Collapsed residual frames from the first epoch using NMF reduction with 1-10 components. The figures share a common scale and the inner FWHM is masked out for all the frames.}
\figsetgrpend

\figsetgrpstart
\figsetgrpnum{11.3}
\figsetgrptitle{2020-02-04 GreeDS mosaic}
\figsetplot{reports/2020feb04_greeds_mosaic}
\figsetgrpnote{Collapsed residual frames from the first epoch using GreeDS reduction with 1-10 components. The figures share a common scale and the inner FWHM is masked out for all the frames.}
\figsetgrpend

%%% second epoch

\figsetgrpstart
\figsetgrpnum{11.4}
\figsetgrptitle{2020-11-21 PCA mosaic}
\figsetplot{reports/2020nov21_pca_mosaic}
\figsetgrpnote{Collapsed residual frames from the second epoch using PCA reduction with 1-10 components. The figures share a common scale and the inner FWHM is masked out for all the frames.}
\figsetgrpend

\figsetgrpstart
\figsetgrpnum{11.5}
\figsetgrptitle{2020-11-21 NMF mosaic}
\figsetplot{reports/2020nov21_nmf_mosaic}
\figsetgrpnote{Collapsed residual frames from the second epoch using NMF reduction with 1-10 components. The figures share a common scale and the inner FWHM is masked out for all the frames.}
\figsetgrpend

\figsetgrpstart
\figsetgrpnum{11.6}
\figsetgrptitle{2020-11-21 GreeDS mosaic}
\figsetplot{reports/2020nov21_greeds_mosaic}
\figsetgrpnote{Collapsed residual frames from the second epoch using GreeDS reduction with 1-10 components. The figures share a common scale and the inner FWHM is masked out for all the frames.}
\figsetgrpend

%%% third epoch

\figsetgrpstart
\figsetgrpnum{11.7}
\figsetgrptitle{2020-11-28 PCA mosaic}
\figsetplot{reports/2020nov28_pca_mosaic}
\figsetgrpnote{Collapsed residual frames from the third epoch using PCA reduction with 1-10 components. The figures share a common scale and the inner FWHM is masked out for all the frames.}
\figsetgrpend

\figsetgrpstart
\figsetgrpnum{11.8}
\figsetgrptitle{2020-11-28 NMF mosaic}
\figsetplot{reports/2020nov28_nmf_mosaic}
\figsetgrpnote{Collapsed residual frames from the third epoch using NMF reduction with 1-10 components. The figures share a common scale and the inner FWHM is masked out for all the frames.}
\figsetgrpend

\figsetgrpstart
\figsetgrpnum{11.9}
\figsetgrptitle{2020-11-28 GreeDS mosaic}
\figsetplot{reports/2020nov28_greeds_mosaic}
\figsetgrpnote{Collapsed residual frames from the third epoch using GreeDS reduction with 1-10 components. The figures share a common scale and the inner FWHM is masked out for all the frames.}
\figsetgrpend

\figsetend

%%%%%%%%%%%%%%%%%%%%%%%%%%%%%%%%%%%%%%%%%%%%%%%%

\begin{figure}[h!]
    \centering
    \figurenum{11}
    \epsscale{0.9}
    \plotone{reports/2020feb04_pca_mosaic}
    \caption{Collapsed residual frames from the first epoch using PCA reduction with 1-10 components. The figures share a common scale and the inner FWHM is masked out for all the frames. Outputs for the other epochs and for the NMF and GreeDS algorithms (9 figures) are in the online figure set and the GitHub repository}
\end{figure}

%%%%%%%%%%%%%%%%%%%%%%%%%%%%%%%%%%%%%%%%%%%%%%%%

\figsetstart
\figsetnum{12}
\figsettitle{PCA, NMF, and GreeDS results}

%%% first epoch

\figsetgrpstart
\figsetgrpnum{12.1}
\figsetgrptitle{2020-02-04 PCA contrast}
\figsetplot{reports/2020feb04_pca_contrast_curves}
\figsetgrpnote{5$\sigma$ Gaussian contrast curves for the first epoch using PCA reduction with 1-10 components. The left two figures are the average STIM probability map, and the SLIM detection map. For both of these maps, a typical cutoff value is 0.5.}
\figsetgrpend

\figsetgrpstart
\figsetgrpnum{12.2}
\figsetgrptitle{2020-02-04 NMF contrast}
\figsetplot{reports/2020feb04_nmf_contrast_curves}
\figsetgrpnote{5$\sigma$ Gaussian contrast curves for the first epoch using NMF reduction with 1-10 components. The left two figures are the average STIM probability map, and the SLIM detection map. For both of these maps, a typical cutoff value is 0.5.}
\figsetgrpend

\figsetgrpstart
\figsetgrpnum{12.3}
\figsetgrptitle{2020-02-04 GreeDS contrast}
\figsetplot{reports/2020feb04_greeds_contrast_curves}
\figsetgrpnote{5$\sigma$ Gaussian contrast curves for the first epoch using GreeDS reduction with 1-10 components. The left two figures are the average STIM probability map, and the SLIM detection map. For both of these maps, a typical cutoff value is 0.5.}
\figsetgrpend

%%% second epoch

\figsetgrpstart
\figsetgrpnum{12.4}
\figsetgrptitle{2020-11-21 PCA contrast}
\figsetplot{reports/2020nov21_pca_contrast_curves}
\figsetgrpnote{5$\sigma$ Gaussian contrast curves for the second epoch using PCA reduction with 1-10 components. The left two figures are the average STIM probability map, and the SLIM detection map. For both of these maps, a typical cutoff value is 0.5.}
\figsetgrpend

\figsetgrpstart
\figsetgrpnum{12.5}
\figsetgrptitle{2020-11-21 NMF contrast}
\figsetplot{reports/2020nov21_nmf_contrast_curves}
\figsetgrpnote{5$\sigma$ Gaussian contrast curves for the second epoch using NMF reduction with 1-10 components. The left two figures are the average STIM probability map, and the SLIM detection map. For both of these maps, a typical cutoff value is 0.5.}
\figsetgrpend

\figsetgrpstart
\figsetgrpnum{12.6}
\figsetgrptitle{2020-11-21 GreeDS contrast}
\figsetplot{reports/2020nov21_greeds_contrast_curves}
\figsetgrpnote{5$\sigma$ Gaussian contrast curves for the second epoch using GreeDS reduction with 1-10 components. The left two figures are the average STIM probability map, and the SLIM detection map. For both of these maps, a typical cutoff value is 0.5.}
\figsetgrpend

%%% third epoch

\figsetgrpstart
\figsetgrpnum{12.7}
\figsetgrptitle{2020-11-28 PCA contrast}
\figsetplot{reports/2020nov28_pca_contrast_curves}
\figsetgrpnote{5$\sigma$ Gaussian contrast curves for the third epoch using PCA reduction with 1-10 components. The left two figures are the average STIM probability map, and the SLIM detection map. For both of these maps, a typical cutoff value is 0.5.}
\figsetgrpend

\figsetgrpstart
\figsetgrpnum{12.8}
\figsetgrptitle{2020-11-28 NMF contrast}
\figsetplot{reports/2020nov28_nmf_contrast_curves}
\figsetgrpnote{5$\sigma$ Gaussian contrast curves for the third epoch using NMF reduction with 1-10 components. The left two figures are the average STIM probability map, and the SLIM detection map. For both of these maps, a typical cutoff value is 0.5.}
\figsetgrpend

\figsetgrpstart
\figsetgrpnum{12.9}
\figsetgrptitle{2020-11-28 GreeDS contrast}
\figsetplot{reports/2020nov28_greeds_contrast_curves}
\figsetgrpnote{5$\sigma$ Gaussian contrast curves for the third epoch using GreeDS reduction with 1-10 components. The left two figures are the average STIM probability map, and the SLIM detection map. For both of these maps, a typical cutoff value is 0.5.}
\figsetgrpend

\figsetend

%%%%%%%%%%%%%%%%%%%%%%%%%%%%%%%%%%%%%%%%%%%%%%%%

\begin{figure}[h!]
    \centering
    \figurenum{12}
    \epsscale{0.9}
    \plotone{reports/2020feb04_pca_contrast_curves}
    \caption{5$\sigma$ Gaussian contrast curves for the first epoch using PCA reduction with 1-10 components. The left two figures are the average STIM probability map, and the SLIM detection map. For both of these maps, a typical cutoff value is 0.5. Outputs for the other epochs and for the NMF and GreeDS algorithms (9 figures) are in the online figure set and the GitHub repository.}
    \label{fig:pca-contrast-curves}
\end{figure}
