Results from processing each epoch of Sirius B data with various ADI algorithms. The top-left image is the subtracted, derotated, and collapsed image. The top-middle image is the Gaussian S/N map, the bottom-middle image is the Student-t S/N (significance) map, which typically requires a value of 5 to show a significant detection, and the bottom-left image is the STIM probability map, which typically requires a probability of 0.5 for significant detection. The right graph shows the 5$\sigma$ contrast, both the Gaussian and Student-t corrected curves.    


\begin{figure}[h!]
    \centering
    \epsscale{0.9}
    \plotone{reports/2020feb04_median}
    \caption{2020-02-04 median}
\end{figure}

\begin{figure}[h!]
    \centering
    \epsscale{0.9}
    \plotone{reports/2020feb04_pca-2}
    \caption{2020-02-04 PCA(2)}
\end{figure}

\begin{figure}[h!]
    \centering
    \epsscale{0.9}
    \plotone{reports/2020feb04_nmf-2}
    \caption{2020-02-04 NMF(2))}
\end{figure}

\begin{figure}[h!]
    \centering
    \epsscale{0.9}
    \plotone{reports/2020feb04_greeds-2}
    \caption{2020-02-04 GreeDS(2)}
\end{figure}

\begin{figure}[h!]
    \centering
    \epsscale{0.9}
    \plotone{reports/2020feb04_annular_median}
    \caption{2020-02-04 annular median}
\end{figure}

\begin{figure}[h!]
    \centering
    \epsscale{0.9}
    \plotone{reports/2020feb04_annular_pca-2}
    \caption{2020-02-04 annular PCA(2))}
\end{figure}

\begin{figure}[h!]
    \centering
    \epsscale{0.9}
    \plotone{reports/2020feb04_annular_nmf-2}
    \caption{2020-02-04 annular NMF(2)}
\end{figure}

\begin{figure}[h!]
    \centering
    \epsscale{0.9}
    \plotone{2020feb04_contrast_curves}
    \caption{5$\sigma$ contrast curves from various ADI algorithms for the 2020-02-04 epoch. Both the Gaussian (solid lines) and Student-t corrected (dashed lines) contrast curves are shown.}
\end{figure}

%%%%%%%%%%%%%%%%%

\begin{figure}[h!]
    \centering
    \epsscale{0.9}
    \plotone{reports/2020nov21_median}
    \caption{2020-11-21 median subtraction.}
\end{figure}

\begin{figure}[h!]
    \centering
    \epsscale{0.9}
    \plotone{reports/2020nov21_pca-2}
    \caption{2020-11-21 PCA subtraction with 2 components.}
\end{figure}

\begin{figure}[h!]
    \centering
    \epsscale{0.9}
    \plotone{reports/2020nov21_nmf-2}
    \caption{2020-11-21 NMF subtraction with 2 components.}
\end{figure}

\begin{figure}[h!]
    \centering
    \epsscale{0.9}
    \plotone{reports/2020nov21_greeds-2}    
    \caption{2020-11-21 GreeDS with 2 components.}
\end{figure}

\begin{figure}[h!]
    \centering
    \epsscale{0.9}
    \plotone{reports/2020nov21_annular_median}
    \caption{2020-11-21 annular median subtraction with a rotation threshold of 0.5.}
\end{figure}

\begin{figure}[h!]
    \centering
    \epsscale{0.9}
    \plotone{reports/2020nov21_annular_pca-2}
    \caption{2020-11-21 annular PCA subtraction with 2 components and a rotation threshold ofnov21 0.5.}
\end{figure}

\begin{figure}[h!]
    \centering
    \epsscale{0.9}
    \plotone{reports/2020nov21_annular_nmf-2}
    \caption{2020-11-21 annular NMF subtraction with 2 components and a rotation threshold of 0.5.}
\end{figure}

\begin{figure}[h!]
    \centering
    \epsscale{0.9}
    \plotone{2020nov21_contrast_curves}
    \caption{5$\sigma$ contrast curves from various ADI algorithms for the 2020-11-21 epoch. Both the Gaussian (solid lines) and Student-t corrected (dashed lines) contrast curves are shown.}
\end{figure}

%%%%%%%%%%%%%%%%%

\begin{figure}[h!]
    \centering
    \epsscale{0.9}
    \plotone{reports/2020nov28_median}
    \caption{2020-11-28 median subtraction.}
\end{figure}

\begin{figure}[h!]
    \centering
    \epsscale{0.9}
    \plotone{reports/2020nov28_pca-2}
    \caption{2020-11-28 PCA subtraction with 2 components.}
\end{figure}

\begin{figure}[h!]
    \centering
    \epsscale{0.9}
    \plotone{reports/2020nov28_nmf-2}
    \caption{2020-11-28 NMF subtraction with 2 components.}
\end{figure}

\begin{figure}[h!]
    \centering
    \epsscale{0.9}
    \plotone{reports/2020nov28_greeds-2}
    \caption{2020-11-28 GreeDS with 2 components.}
\end{figure}

\begin{figure}[h!]
    \centering
    \epsscale{0.9}
    \plotone{reports/2020nov28_annular_median}
    \caption{2020-11-28 annular median subtraction with a rotation threshold of 0.5.}
\end{figure}

\begin{figure}[h!]
    \centering
    \epsscale{0.9}
    \plotone{reports/2020nov28_annular_pca-2}
    \caption{2020-11-28 annular PCA subtraction with 2 components and a rotation threshold ofnov28 0.5.}
\end{figure}

\begin{figure}[h!]
    \centering
    \epsscale{0.9}
    \plotone{reports/2020nov28_annular_nmf-2}
    \caption{2020-11-28 annular NMF subtraction with 2 components and a rotation threshold of 0.5.}
\end{figure}

\begin{figure}[h!]
    \centering
    \epsscale{0.9}
    \plotone{2020nov28_contrast_curves}
    \caption{5$\sigma$ contrast curves from various ADI algorithms for the 2020-11-28 epoch. Both the Gaussian (solid lines) and Student-t corrected (dashed lines) contrast curves are shown.}
\end{figure}
