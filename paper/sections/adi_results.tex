
\figsetstart
\figsetnum{6}
\figsettitle{ADI processing results}

\figsetgrpstart
\figsetgrpnum{6.1}
\figsetgrptitle{2020-02-04 median}
\figsetplot{reports/2020feb04_median}
\figsetgrpnote{Results from processing each epoch of Sirius B data with various ADI algorithms. The top-left image is the subtracted, derotated, and collapsed image. The top-middle image is the Gaussian S/N map, the bottom-middle image is the Student-t S/N (significance) map, which typically requires a value of 5 to show a significant detection, and the bottom-left image is the STIM probability map, which typically requires a probability of 0.5 for significant detection. The right graph shows the 5$\sigma$ contrast, both the Gaussian and Student-t corrected curves.}
\figsetgrpend

\figsetgrpstart
\figsetgrpnum{6.2}
\figsetgrptitle{2020-02-04 PCA(2)}
\figsetplot{reports/2020feb04_pca-2}
\figsetgrpnote{Results from processing each epoch of Sirius B data with various ADI algorithms. The top-left image is the subtracted, derotated, and collapsed image. The top-middle image is the Gaussian S/N map, the bottom-middle image is the Student-t S/N (significance) map, which typically requires a value of 5 to show a significant detection, and the bottom-left image is the STIM probability map, which typically requires a probability of 0.5 for significant detection. The right graph shows the 5$\sigma$ contrast, both the Gaussian and Student-t corrected curves.}
\figsetgrpend

\figsetgrpstart
\figsetgrpnum{6.3}
\figsetgrptitle{2020-02-04 NMF(2))}
\figsetplot{reports/2020feb04_nmf-2}
\figsetgrpnote{Results from processing each epoch of Sirius B data with various ADI algorithms. The top-left image is the subtracted, derotated, and collapsed image. The top-middle image is the Gaussian S/N map, the bottom-middle image is the Student-t S/N (significance) map, which typically requires a value of 5 to show a significant detection, and the bottom-left image is the STIM probability map, which typically requires a probability of 0.5 for significant detection. The right graph shows the 5$\sigma$ contrast, both the Gaussian and Student-t corrected curves.}
\figsetgrpend

\figsetgrpstart
\figsetgrpnum{6.4}
\figsetgrptitle{2020-02-04 GreeDS(2)}
\figsetplot{reports/2020feb04_greeds-2}
\figsetgrpnote{Results from processing each epoch of Sirius B data with various ADI algorithms. The top-left image is the subtracted, derotated, and collapsed image. The top-middle image is the Gaussian S/N map, the bottom-middle image is the Student-t S/N (significance) map, which typically requires a value of 5 to show a significant detection, and the bottom-left image is the STIM probability map, which typically requires a probability of 0.5 for significant detection. The right graph shows the 5$\sigma$ contrast, both the Gaussian and Student-t corrected curves.}
\figsetgrpend

\figsetgrpstart
\figsetgrpnum{6.5}
\figsetgrptitle{2020-02-04 annular median}
\figsetplot{reports/2020feb04_annular_median}
\figsetgrpnote{Results from processing each epoch of Sirius B data with various ADI algorithms. The top-left image is the subtracted, derotated, and collapsed image. The top-middle image is the Gaussian S/N map, the bottom-middle image is the Student-t S/N (significance) map, which typically requires a value of 5 to show a significant detection, and the bottom-left image is the STIM probability map, which typically requires a probability of 0.5 for significant detection. The right graph shows the 5$\sigma$ contrast, both the Gaussian and Student-t corrected curves.}
\figsetgrpend

\figsetgrpstart
\figsetgrpnum{6.6}
\figsetgrptitle{2020-02-04 annular PCA(2))}
\figsetplot{reports/2020feb04_annular_pca-2}
\figsetgrpnote{Results from processing each epoch of Sirius B data with various ADI algorithms. The top-left image is the subtracted, derotated, and collapsed image. The top-middle image is the Gaussian S/N map, the bottom-middle image is the Student-t S/N (significance) map, which typically requires a value of 5 to show a significant detection, and the bottom-left image is the STIM probability map, which typically requires a probability of 0.5 for significant detection. The right graph shows the 5$\sigma$ contrast, both the Gaussian and Student-t corrected curves.}
\figsetgrpend

\figsetgrpstart
\figsetgrpnum{6.7}
\figsetgrptitle{2020-02-04 annular NMF(2)}
\figsetplot{reports/2020feb04_annular_nmf-2}
\figsetgrpnote{Results from processing each epoch of Sirius B data with various ADI algorithms. The top-left image is the subtracted, derotated, and collapsed image. The top-middle image is the Gaussian S/N map, the bottom-middle image is the Student-t S/N (significance) map, which typically requires a value of 5 to show a significant detection, and the bottom-left image is the STIM probability map, which typically requires a probability of 0.5 for significant detection. The right graph shows the 5$\sigma$ contrast, both the Gaussian and Student-t corrected curves.}
\figsetgrpend

\figsetgrpstart
\figsetgrpnum{6.8}
\figsetgrptitle{2020-11-21 median subtraction.}
\figsetplot{reports/2020nov21_median}
\figsetgrpnote{Results from processing each epoch of Sirius B data with various ADI algorithms. The top-left image is the subtracted, derotated, and collapsed image. The top-middle image is the Gaussian S/N map, the bottom-middle image is the Student-t S/N (significance) map, which typically requires a value of 5 to show a significant detection, and the bottom-left image is the STIM probability map, which typically requires a probability of 0.5 for significant detection. The right graph shows the 5$\sigma$ contrast, both the Gaussian and Student-t corrected curves.}
\figsetgrpend

\figsetgrpstart
\figsetgrpnum{6.9}
\figsetgrptitle{2020-11-21 PCA subtraction with 2 components.}
\figsetplot{reports/2020nov21_pca-2}
\figsetgrpnote{Results from processing each epoch of Sirius B data with various ADI algorithms. The top-left image is the subtracted, derotated, and collapsed image. The top-middle image is the Gaussian S/N map, the bottom-middle image is the Student-t S/N (significance) map, which typically requires a value of 5 to show a significant detection, and the bottom-left image is the STIM probability map, which typically requires a probability of 0.5 for significant detection. The right graph shows the 5$\sigma$ contrast, both the Gaussian and Student-t corrected curves.}
\figsetgrpend

\figsetgrpstart
\figsetgrpnum{6.10}
\figsetgrptitle{2020-11-21 NMF subtraction with 2 components.}
\figsetplot{reports/2020nov21_nmf-2}
\figsetgrpnote{Results from processing each epoch of Sirius B data with various ADI algorithms. The top-left image is the subtracted, derotated, and collapsed image. The top-middle image is the Gaussian S/N map, the bottom-middle image is the Student-t S/N (significance) map, which typically requires a value of 5 to show a significant detection, and the bottom-left image is the STIM probability map, which typically requires a probability of 0.5 for significant detection. The right graph shows the 5$\sigma$ contrast, both the Gaussian and Student-t corrected curves.}
\figsetgrpend

\figsetgrpstart
\figsetgrpnum{6.11}
\figsetgrptitle{2020-11-21 GreeDS with 2 components.}
\figsetplot{reports/2020nov21_greeds-2}
\figsetgrpnote{Results from processing each epoch of Sirius B data with various ADI algorithms. The top-left image is the subtracted, derotated, and collapsed image. The top-middle image is the Gaussian S/N map, the bottom-middle image is the Student-t S/N (significance) map, which typically requires a value of 5 to show a significant detection, and the bottom-left image is the STIM probability map, which typically requires a probability of 0.5 for significant detection. The right graph shows the 5$\sigma$ contrast, both the Gaussian and Student-t corrected curves.}
\figsetgrpend

\figsetgrpstart
\figsetgrpnum{6.12}
\figsetgrptitle{2020-11-21 annular median subtraction with a rotation threshold of 0.5.}
\figsetplot{reports/2020nov21_annular_median}
\figsetgrpnote{Results from processing each epoch of Sirius B data with various ADI algorithms. The top-left image is the subtracted, derotated, and collapsed image. The top-middle image is the Gaussian S/N map, the bottom-middle image is the Student-t S/N (significance) map, which typically requires a value of 5 to show a significant detection, and the bottom-left image is the STIM probability map, which typically requires a probability of 0.5 for significant detection. The right graph shows the 5$\sigma$ contrast, both the Gaussian and Student-t corrected curves.}
\figsetgrpend

\figsetgrpstart
\figsetgrpnum{6.13}
\figsetgrptitle{2020-11-21 annular PCA subtraction with 2 components and a rotation threshold ofnov21 0.5.}
\figsetplot{reports/2020nov21_annular_pca-2}
\figsetgrpnote{Results from processing each epoch of Sirius B data with various ADI algorithms. The top-left image is the subtracted, derotated, and collapsed image. The top-middle image is the Gaussian S/N map, the bottom-middle image is the Student-t S/N (significance) map, which typically requires a value of 5 to show a significant detection, and the bottom-left image is the STIM probability map, which typically requires a probability of 0.5 for significant detection. The right graph shows the 5$\sigma$ contrast, both the Gaussian and Student-t corrected curves.}
\figsetgrpend

\figsetgrpstart
\figsetgrpnum{6.14}
\figsetgrptitle{2020-11-21 annular NMF subtraction with 2 components and a rotation threshold of 0.5.}
\figsetplot{reports/2020nov21_annular_nmf-2}
\figsetgrpnote{Results from processing each epoch of Sirius B data with various ADI algorithms. The top-left image is the subtracted, derotated, and collapsed image. The top-middle image is the Gaussian S/N map, the bottom-middle image is the Student-t S/N (significance) map, which typically requires a value of 5 to show a significant detection, and the bottom-left image is the STIM probability map, which typically requires a probability of 0.5 for significant detection. The right graph shows the 5$\sigma$ contrast, both the Gaussian and Student-t corrected curves.}
\figsetgrpend

\figsetgrpstart
\figsetgrpnum{6.15}
\figsetgrptitle{2020-11-28 median subtraction.}
\figsetplot{reports/2020nov28_median}
\figsetgrpnote{Results from processing each epoch of Sirius B data with various ADI algorithms. The top-left image is the subtracted, derotated, and collapsed image. The top-middle image is the Gaussian S/N map, the bottom-middle image is the Student-t S/N (significance) map, which typically requires a value of 5 to show a significant detection, and the bottom-left image is the STIM probability map, which typically requires a probability of 0.5 for significant detection. The right graph shows the 5$\sigma$ contrast, both the Gaussian and Student-t corrected curves.}
\figsetgrpend

\figsetgrpstart
\figsetgrpnum{6.16}
\figsetgrptitle{2020-11-28 PCA subtraction with 2 components.}
\figsetplot{reports/2020nov28_pca-2}
\figsetgrpnote{Results from processing each epoch of Sirius B data with various ADI algorithms. The top-left image is the subtracted, derotated, and collapsed image. The top-middle image is the Gaussian S/N map, the bottom-middle image is the Student-t S/N (significance) map, which typically requires a value of 5 to show a significant detection, and the bottom-left image is the STIM probability map, which typically requires a probability of 0.5 for significant detection. The right graph shows the 5$\sigma$ contrast, both the Gaussian and Student-t corrected curves.}
\figsetgrpend

\figsetgrpstart
\figsetgrpnum{6.17}
\figsetgrptitle{2020-11-28 NMF subtraction with 2 components.}
\figsetplot{reports/2020nov28_nmf-2}
\figsetgrpnote{Results from processing each epoch of Sirius B data with various ADI algorithms. The top-left image is the subtracted, derotated, and collapsed image. The top-middle image is the Gaussian S/N map, the bottom-middle image is the Student-t S/N (significance) map, which typically requires a value of 5 to show a significant detection, and the bottom-left image is the STIM probability map, which typically requires a probability of 0.5 for significant detection. The right graph shows the 5$\sigma$ contrast, both the Gaussian and Student-t corrected curves.}
\figsetgrpend

\figsetgrpstart
\figsetgrpnum{6.18}
\figsetgrptitle{2020-11-28 GreeDS with 2 components.}
\figsetplot{reports/2020nov28_greeds-2}
\figsetgrpnote{Results from processing each epoch of Sirius B data with various ADI algorithms. The top-left image is the subtracted, derotated, and collapsed image. The top-middle image is the Gaussian S/N map, the bottom-middle image is the Student-t S/N (significance) map, which typically requires a value of 5 to show a significant detection, and the bottom-left image is the STIM probability map, which typically requires a probability of 0.5 for significant detection. The right graph shows the 5$\sigma$ contrast, both the Gaussian and Student-t corrected curves.}
\figsetgrpend

\figsetgrpstart
\figsetgrpnum{6.19}
\figsetgrptitle{2020-11-28 annular median subtraction with a rotation threshold of 0.5.}
\figsetplot{reports/2020nov28_annular_median}
\figsetgrpnote{Results from processing each epoch of Sirius B data with various ADI algorithms. The top-left image is the subtracted, derotated, and collapsed image. The top-middle image is the Gaussian S/N map, the bottom-middle image is the Student-t S/N (significance) map, which typically requires a value of 5 to show a significant detection, and the bottom-left image is the STIM probability map, which typically requires a probability of 0.5 for significant detection. The right graph shows the 5$\sigma$ contrast, both the Gaussian and Student-t corrected curves.}
\figsetgrpend

\figsetgrpstart
\figsetgrpnum{6.20}
\figsetgrptitle{2020-11-28 annular PCA subtraction with 2 components and a rotation threshold ofnov28 0.5.}
\figsetplot{reports/2020nov28_annular_pca-2}
\figsetgrpnote{Results from processing each epoch of Sirius B data with various ADI algorithms. The top-left image is the subtracted, derotated, and collapsed image. The top-middle image is the Gaussian S/N map, the bottom-middle image is the Student-t S/N (significance) map, which typically requires a value of 5 to show a significant detection, and the bottom-left image is the STIM probability map, which typically requires a probability of 0.5 for significant detection. The right graph shows the 5$\sigma$ contrast, both the Gaussian and Student-t corrected curves.}
\figsetgrpend

\figsetgrpstart
\figsetgrpnum{6.21}
\figsetgrptitle{2020-11-28 annular NMF subtraction with 2 components and a rotation threshold of 0.5.}
\figsetplot{reports/2020nov28_annular_nmf-2}
\figsetgrpnote{Results from processing each epoch of Sirius B data with various ADI algorithms. The top-left image is the subtracted, derotated, and collapsed image. The top-middle image is the Gaussian S/N map, the bottom-middle image is the Student-t S/N (significance) map, which typically requires a value of 5 to show a significant detection, and the bottom-left image is the STIM probability map, which typically requires a probability of 0.5 for significant detection. The right graph shows the 5$\sigma$ contrast, both the Gaussian and Student-t corrected curves.}
\figsetgrpend

\figsetend

\begin{figure*}
\figurenum{6}
\plotone{reports/2020feb04_median}
\caption{Results from processing each epoch of Sirius B data with various ADI algorithms. The top-left image is the subtracted, derotated, and collapsed image. The top-middle image is the Gaussian S/N map, the bottom-middle image is the Student-t S/N (significance) map, which typically requires a value of 5 to show a significant detection, and the bottom-left image is the STIM probability map, which typically requires a probability of 0.5 for significant detection. The right graph shows the 5$\sigma$ contrast, both the Gaussian and Student-t corrected curves. The online figure set (21 images) contains the outputs for each algorithm tested on each epoch.}
\end{figure*}
