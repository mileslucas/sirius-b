\documentclass[twocolumn,linenumbers]{aastex631}
\usepackage{cleveref}
\let\tablenum\relax
\usepackage{siunitx}
\usepackage{makecell}

% workaround for cleveref+AASTeX
% https://github.com/AASJournals/AASTeX60/issues/69#issuecomment-685852891
\makeatletter
\usepackage{etoolbox}
\patchcmd\H@refstepcounter{\protected@edef}{\protected@xdef}{}{}
\makeatother

% custom SI units
\DeclareSIUnit\parsec{pc}
\DeclareSIUnit\au{AU}
\DeclareSIUnit\jupitermass{M_J}
\DeclareSIUnit\milliarcsecond{mas}
\DeclareSIUnit\solarmass{M_\odot}
\DeclareSIUnit\year{yr}
\DeclareSIUnit\pixel{px}
\sisetup{
    list-units=single,
    separate-uncertainty
}

%% Define new commands here
\newcommand\latex{La\TeX}

\graphicspath{{./}{figures/}}
\shortauthors{Lucas et al.}

\begin{document}

\title{An Imaging Search for Circumstellar Companions of Sirius B}

\shorttitle{An Imaging Search for Circumstellar Companions of Sirius B}
\shortauthors{Lucas et al.}

\correspondingauthor{Miles Lucas}
\email{mdlucas@hawaii.edu}

\author[0000-0001-6341-310X]{Miles Lucas}
\affiliation{Institute for Astronomy, University of Hawaii at Manoa, 2680 Woodlawn Dr, Honolulu, HI 96822, USA}

\author[0000-0003-1341-5531]{Michael Bottom}
\affiliation{Institute for Astronomy, University of Hawaii at Hilo, 640 N Aohoku Pl, Hilo, HI 96720, USA}

\author[0000-0003-4769-1665]{Garreth Ruane}
\affiliation{Jet Propulsion Laboratory, California Institute of Technology, 4800 Oak Grove Dr, Pasadena, CA 91109, USA}

\author[0000-0002-0696-1780]{Sam Ragland}
\affiliation{W.M. Keck Observatory, 65-1120 Mamalahoa Hwy, Waimea, HI 96743, USA}


\begin{abstract}
We present deep imaging of Sirius B, the closest and brightest white dwarf. We use Keck/NIRC2 in Lp-band (\qty{3.776}{\micro\meter}) across three epochs in 2020 using the technique of angular differential imaging. We reach sub-Jupiter sensitivities and sub-AU separations, reaching \qty{3.5}{\jupitermass} at \qty{0.25}{\au} down to a sensitivity of \qty{0.6}{\jupitermass} at \qty{1.5}{\au}. The uncertainty in mass sensitivity is below \qty{0.1}{\jupitermass} due to the precisely known Sirius B system age. Consistent with previous results, we do not detect any companions around Sirius B.
\end{abstract}

\section{Introduction} \label{sec:intro}

High-contrast imaging (HCI) is a powerful technique for discovering and characterizing exoplanets. Probing the architecture, formation, and atmospheres of planets directly is necessary for advancing substellar companion formation and evolution theory. Imaging a planet is challenging, however. The typical astrophysical flux ratios (contrast) for a Sun-Jupiter analog in the near-infrared (NIR) are $\sim$$10^{\text{-}8}$, and for a Sun-Earth system are $\sim$$10^{\text{-}10}$ \citep{traubDirectImagingExoplanets2010}. Thermal emission from exoplanets peaks in the infrared and is well into the Rayleigh-Jeans limit of the star, decreasing contrast compared to the visible or ultraviolet. In addition, the close angular separations of planets make it difficult to detect them over the diffraction pattern of their host and other noise sources.

White dwarfs (WD) are interesting targets for direct imaging- they are \numrange{3}{4} orders of magnitude fainter than their main-sequence (MS) progenitors which reduces the contrast necessary to detect a companion. In addition, the relative lack of spectral features of WDs makes them imprecise targets for radial velocity analysis. The intrinsic faintness of WDs proves challenging for all detection methods; this is exacerbated on ground-based telescopes which require significant signal-to-noise (S/N) to operate their adaptive optics (AO) facilities. This implicitly limits searches of WDs to the $\sim$\qty{100}{\parsec} neighborhood. Little is known about post-MS planetary evolution, however, planets that are tenuous enough to survive the red giant and asymptotic giant phases of its host are interesting laboratories for planetary formation pathways as showcased by recent discoveries like giant planet candidate WD 1856+534b \citep{vanderburgGiantPlanetCandidate2020}.

In this work, we briefly review post-MS planetary formation theory (\cref{sec:post-ms}) and discuss the Sirius system, specifically bright WD Sirius B, as potential exoplanet hosts (\cref{sec:sirius}). In \cref{sec:obs}, \cref{sec:analysis}, and \cref{sec:results} we describe our observations of Sirius B in 2020, along with the subsequent analysis. Finally, in \cref{sec:conclusion} we discuss the implications of our observations for planetary formation theory and future work.

\section{Post-MS Evolution} \label{sec:post-ms}

There is limited knowledge of planetary systems around evolved stars. \citet{burleighImagingPlanetsNearby2002a,verasPostmainsequencePlanetarySystem2016} suggest exoplanets on initially wide (\textgreater\qty{5}{\au}) orbits around intermediate-mass stars will survive expansion during the RGB phase. Planets very close to the Roche limit of the expanding red giant can still be shredded by tidal forces despite escaping engulfment, although the tidal forces are negligible on planets that are far from the Roche limit (and readily escape engulfment) \citep{nordhausOrbitsLowmassCompanions2013}. During the AGB phase, stellar mass loss adiabatically expands the orbit of the planet by a maximum factor of $M_{*,\mathrm{MS}}/M_{*,\mathrm{WD}}$ \citep{jeansCosmogonicProblemsAssociated1924}. In addition to \textit{first-generation} planets, there are potential methods for \textit{second-generation} planets to form after the violent RGB and AGB phases \citep{peretsSecondGenerationPlanets2010}.

The first search for substellar companions around white dwarfs was conducted by \citet{probstInfraredSearchVery1983}. They searched for infrared (IR) excess in their spectral energy distributions (SED) using broadband photometry. The IR excess would be indicative of thermal emission of faint material around the star, such as a circumstellar disk or giant planet. They studied $\sim$100 white dwarfs but found no companions. The same method was applied by \citet{zuckermanAncientPlanetarySystems2010}, who found excess IR emission around white dwarf G29-38. The emission was determined to be from a dust disk \citep{telescoObservationsG29381990}, which was associated with accretion onto the white dwarf's surface, polluting the stellar atmosphere \citep{koesterMetalsVariableG29381997}. It is estimated that while a third of white dwarfs are polluted, only a few percent have IR excesses \citep{bonsorInfraredObservationsWhite2017}. \citet{skemerSiriusImagedMidinfrared2011} suggested stellar radiation from stars hotter than \qty{15,000}{\kelvin} would sublimate dust, therefore, suppressing any IR excess.

Imaging is well-suited for studying evolved planetary systems, yet to date, only one planet has been imaged around a white dwarf. \citet{luhmanDiscoveryCandidateCoolest2011} discovered a companion around WD 0806-661B with a mass of \qty{7}{\jupitermass} on a wide \qty{2500}{\au} orbit using Spitzer. The ``Degenerate Objects around Degenerate Objects'' survey \citep[DODO;][]{hoganDODOSurveyII2009} observed 29 white dwarfs with Gemini/NIRI and VLT/ISAAC, reaching an average upper limit around $\sim$\qty{8}{\jupitermass} beyond \qty{35}{\au}. The young white dwarf GD 50 was observed using the extreme AO instrument SPHERE at the VLT \citep{xuExtremeAOSearchGiant2015a}, reaching sensitivity limits of \qty{4}{\jupitermass} at \qty{6.2}{\au}.

\section{Sirius B and the Sirius System} \label{sec:sirius}

A particularly fascinating target is Sirius B, the closest and brightest white dwarf. The Sirius system is the 7th closest to the sun at \qty{2.6}{\parsec}, consisting of Sirius A, a -1.35 magnitude A1Vm star known for being the brightest, and Sirius B, a DA2 white dwarf with a \qty{50}{\year} orbit \citep{collaborationGaiaMission2016, bondSiriusSystemIts2017,collaborationGaiaEarlyData2021a}. As mentioned previously, the proximity and faintness of Sirius B (compared to a MS star) make it compelling for imaging. Additionally, the young system age ($\sim$\qty{225}{\mega\year}) means any giant planets would still retain much of their latent formation heat, increasing their luminosity in the IR \citep{fortneyGiantPlanetInterior2010}.

A common problem in direct imaging is determining planetary masses from photometry using planetary atmosphere evolution grids. This method of interpolation depends on the stellar age, the stellar photometry, and the distance to the system. Determining stellar ages is quite difficult and sometimes impossible, and is often the largest source of uncertainty when interpolating planetary masses. Age uncertainty exponentially affects young systems (\textless\qty{1}{\giga\year}) due to the rapid cooling of giant planets. Sirius B avoids these pitfalls due to its unusually high age precision, primarily derived from accurate dynamical masses.

The dynamical masses are determined through astrometric studies; the first study of Sirius was performed by \citet{besselVariationsProperMotions1844}, who recognized wobbles in the proper motion of Sirius A caused by a ``dark satellite''. This dark satellite was visually confirmed in \citet{bondCompanionSiriusProf1862} as Sirius B. \citet{adamsSpectrumCompanionSirius1915} took the first spectral measurements of Sirius B and found it to be similar to a MS early A-type star, despite its faintness, which we now know to be typical of white dwarf spectra.

Initial astrometric measurements suggested a 50 year orbital period for Sirius B \citep{auwersOrbitSirius1864}. \citet{vandenbosOrbitSiriusADS1960,gatewoodStudySirius1978} were the first to estimate dynamical masses using compilations of photographic plates from Lick and Yerkes observatories. \citet{bondSiriusSystemIts2017} greatly refined the orbital solution using a compilation of historical data and Hubble Space Telescope (HST) data, which gave dynamical masses of \qty{2.063+-0.023}{\solarmass} and \qty{1.018+-0.011}{\solarmass} for A and B, respectively. A companion around Sirius B would be affected by the orbit of Sirius A, and this constrained three-body system has been studied numerically \citep{holmanLongTermStabilityPlanets1999}. \citet{bondSiriusSystemIts2017} calculate the longest period stable companion around Sirius B is \qty{1.79}{\year}, which corresponds to a \qty{1.5}{\au} circular orbit.

To find the total age of Sirius B, \citet{bondSiriusSystemIts2017} used isochrones to constrain the cooling age (\qty{126}{\mega\year}), first. Applying the initial-final mass relation (IFMR) of white dwarfs \citep{cummingsTwoMassiveWhite2016} the estimated progenitor mass of Sirius B is between \qtyrange{5}{5.6}{\solarmass}, which, when combined with stellar evolution codes, yielded total system ages between \qtyrange{226}{228}{\mega\year} with an uncertainty of about \qty{+-10}{\mega\year} \citep{bondSiriusSystemIts2017}. An age uncertainty of $\sim$1\% is exceptional compared to the $\sim$10\% or worse of stars found in young moving groups, not to mention many stars cannot be precisely dated at all.

The first modern imaging study searching for companions around Sirius B was \citet{schroederSearchFaintCompanions2000} who used the HST wide-field planetary camera (WFPC) at \qty{1}{\micro\meter}. Around the same time \citet{kuchnerSearchExozodiacalDust2000} searched in a narrower field of view (FOV) with HST/NICMOS at \qty{1}{\micro\meter}. These studies combined had a sensitivity down to $\sim$\qty{10}{\jupitermass} at \qty{5.3}{\au} (\ang{;;2}). \citet{bonnet-bidaudADONISHighContrast2008a} used the ground-based ESO/ADONIS instrument in J, H, and Ks-band and reached a sensitivty of $\sim$\qty{30}{\jupitermass} at \qty{7.9}{\au} (\ang{;;3}). \citet{skemerSiriusImagedMidinfrared2011} used mid-IR (up to \qty{10}{\micro\meter}) observations from Gemini/T-ReCs which ruled out evidence for any infrared excess around Sirius B. \citet{thalmannPiercingGlareDirect2011} used Subaru/IRCS at \qty{4.05}{\micro\meter} reaching detection sensitivities of \qtyrange{6}{12}{\jupitermass} at \ang{;;1}. Recently, \citet{pathakHighContrastImaging2021} took simultaneous mid-IR observations (\qty{10}{\micro\meter}) at VLT/VISIR of Sirius A (through a coronagraph) and B. Because of the simultaneous observation, their contrast depended on which region of the FOV was tested. Their average sensitivity is $\sim$\qty{2.5}{\jupitermass} at \qty{1}{\au}, and their best sensitivity (from the ``inner'' region) is $\sim$\qty{1.5}{\jupitermass} at \qty{1}{\au}.

In this work, we report direct images of Sirius B with Keck/NIRC2. The rest of the manuscript is organized as follows: \cref{sec:obs} describes our target and observing strategy, \cref{sec:analysis} describes the processing and analysis techniques used, and \cref{sec:results} describes our results.

\section{Observations} \label{sec:obs}
\defcitealias{bonnet-bidaudADONISHighContrast2008a}{BB08}
\defcitealias{bondSiriusSystemIts2017}{B17}
\defcitealias{collaborationGaiaEarlyData2021a}{G21a}
\defcitealias{cummingsWhiteDwarfInitialFinal2018}{C18}
\begin{deluxetable}{cccc}
    \tablecaption{Parameters of the Sirius system adopted in this study.\label{tbl:system}}
    \tablehead{
        \colhead{parameter} & \colhead{value} & \colhead{unit} & \colhead{ref.}
    }
    \startdata
    $t_\mathrm{sys}$ & 226 & \unit{\mega\year} & \citetalias{bondSiriusSystemIts2017,cummingsWhiteDwarfInitialFinal2018} \\
    $\pi$ & \num{374.5\pm0.23} & mas & \citetalias{collaborationGaiaEarlyData2021a} \\
    $d$ & \num{2.67\pm.0016} & pc & \citetalias{collaborationGaiaEarlyData2021a} \\
    \cutinhead{Sirius A}
    $M_\star$ & \num{2.063\pm0.023} & \unit{\solarmass} & \citetalias{bondSiriusSystemIts2017} \\
    \cutinhead{Sirius B}
    $M_\star$ & \num{1.018\pm0.011} & \unit{\solarmass} & \citetalias{bondSiriusSystemIts2017} \\
    $t_\mathrm{WD}$ & 125 & \unit{\mega\year} & \citetalias{bondSiriusSystemIts2017,cummingsWhiteDwarfInitialFinal2018}\\
    $m^{Lp}$ & \num{9.01\pm0.16} & & \citetalias{bonnet-bidaudADONISHighContrast2008a} \\
    $M^{Lp}$ & \num{11.88\pm0.16} & & \citetalias{bonnet-bidaudADONISHighContrast2008a,collaborationGaiaEarlyData2021a} \\
    \enddata
\end{deluxetable}

\begin{deluxetable*}{ccccccccc}[t]
    \tabletypesize{\small}
    \tablecaption{Observing parameters for the three epochs of data. All observations were carried out using the NIRC2 narrow camera (\qty{10}{\milliarcsecond\per\pixel}; \ang{;;2.5}$\times$\ang{;;2.5}) in Lp-band (\qty{3.776}{\micro\meter}). Observation time is based on the frames that were selected for processing. Seeing values were measured at \qty{0.5}{\micro\meter} using a differential image motion monitor (DIMM) and averaged over the observing session. Seeing values, temperature, and water vapor measurements were all taken from the Maunakea weather center forecast archive.
    \label{tbl:obs}}
    \tablehead{
        \colhead{\makecell{Date\\observed}} &
        \colhead{\makecell{Sirius B\\offset} (\unit{\arcsecond})} &
        \colhead{\makecell{Sirius B\\PA} (\unit{\degree})} &
        \colhead{\makecell{Obs.\\time} (hr)} &
        \colhead{\makecell{FOV\\rotation} (\unit{\degree})} &
        \colhead{FWHM (\unit{\milliarcsecond})} &
        \colhead{Seeing (\unit{\arcsecond})} &
        \colhead{Temp (\unit{\celsius})} &
        \colhead{PWV (\unit{\milli\meter})}
    }
    \startdata
    2020-02-04 & 11.20 & 67.90 & 1.44 & 60.1 & 79.9 & 0.936 & 0.0 & 0.7 \\
    2020-11-21 & 11.27 & 66.42 & 2.91 & 91.4 & 76.4 & 0.871 & 0.8 & 3.5 \\
    2020-11-28 & 11.27 & 66.38 & 2.44 & 80.4 & 82.2 & 1.23 & -1.5 & 3.0 \\
    \enddata
\end{deluxetable*}


% \begin{deluxetable}{ccccc}
%     \tabletypesize{\small}
%     \tablecaption{Observing parameters for the three epochs of data. All observations were carried out using the NIRC2 narrow camera (\qty{10}{\milliarcsecond\per\pixel}; \ang{;;2.5}$\times$\ang{;;2.5}) in Lp-band (\qty{3.776}{\micro\meter}). Observation time is based on the frames that were selected for processing. Seeing values were measured at \qty{0.5}{\micro\meter} using a differential image motion monitor (DIMM) and averaged over the observing session. Seeing values, temperature, and water vapor measurements were all taken from the Maunakea weather center forecast archive.
%     \label{tbl:obs}}
%     \tablehead{
%         \colhead{Epoch} &
%         \colhead{2020-02-4} &
%         \colhead{2020-11-21} &
%         \colhead{2020-11-28} &
%         \colhead{}
%     }
%     \startdata
%     Sirius B offset & 11.20 & 11.27 & 11.27 & \unit{\arcsecond}  \\
%     Sirius B PA & 67.90 & 66.42 & 66.38 & \unit{\degree} \\
%     Obs. time & 1.44 & 2.91 & 2.44 & hr \\
%     FOV rotation & 60.1 & 91.4 & 80.4 & \unit{\degree} \\
%     FWHM & 79.9 & 76.4 & 82.2 & \unit{\milliarcsecond} \\
%     Seeing & 0.936 & 0.871 & 1.23 & \unit{\arcsecond} \\
%     Temp & 0.0 & 0.8 & -1.5 & \unit{\celsius} \\
%     PWV & 0.7 & 3.5 & 3.0 & \unit{\milli\meter}
%     \enddata
% \end{deluxetable}

\begin{figure}
    \centering
    \epsscale{1.1}
    \plotone{spike}
    \caption{A diffraction spike sweeping across a calibrated science frame of Sirius B from the first epoch. This shows the strong scattered light effects of Sirius A and how they affect the FOV of Sirius B \ang{;;11} away.}
    \label{fig:spike}
\end{figure}

\begin{figure}
    \centering
    \epsscale{1.1}
    \plotone{psf}
    \caption{The median frame from the second epoch showing the instrumental PSF. The inner core has a FWHM of $\sim$\qty{76}{\milliarcsecond}. The speckle pattern is shown in the blobs surrounding the first ring, with roughly 6-way radial symmetry corresponding to the hexagonal shape of the primary mirror.}
    \label{fig:psf}
\end{figure}

Despite Sirius B being the brightest white dwarf in the sky, it is still 10 magnitudes fainter than Sirius A, making it a technically challenging target, especially on ground-based telescopes. We directly imaged Sirius B with Keck/NIRC2 in Lp-band (\qty{3.776}{\micro\meter}) using the narrow camera (\qty{10}{\milliarcsecond\per\pixel}; \ang{;;2.5}$\times$\ang{;;2.5}) across three epochs in 2020 (\cref{tbl:obs}). Our first attempt to observe Sirius B failed due to the strong scattered light from Sirius A. The adaptive optics (AO) calibration failed when the scattered light from Sirius A would sweep into the FOV of the wavefront sensor (WFS). Similarly, trying to deploy the focal-plane vortex coronagraph \citep{serabynKeckObservatoryInfrared2017} failed when the coronagraphic pointing control algorithm, \texttt{QACITS} \citep{hubyOnskyPerformanceQACITS2017a}, performed erratically in the presence of the scattered light. We did not try coronagraphy for the remaining observations. \citet[\S2]{viganHighcontrastImagingSirius2015} reported similar issues in their attempts to image Sirius B coronagraphically using VLT/SPHERE. In order to overcome these obstacles, we decided to try using Sirius A as the AO guide star and off-axis guiding to Sirius B.

The Keck AO system \citep{wizinowichPerformanceKeckObservatory2000} was saturated by Sirius A, so we attenuated the flux using a narrow laser-line filter in the WFS. While still bright (appearing like a $\sim$5 magnitude star on the WFS), this was enough attenuation to close the AO loop. From here, we slewed off-axis using the separations and position angles calculated in \cref{tbl:obs} from the orbital parameters in \citet{bondSiriusSystemIts2017}. In this mode, we noticed higher than usual drift in the focal plane, requiring manually recentering the target every 5 or 10 minutes.

During each observation, we took dark frames and sky-flat frames for calibration. All observations were taken with the telescope's field rotator set to track the telescope pupil in order to exploit the natural rotation of the sky via angular differential imaging \citep[ADI;][]{maroisAngularDifferentialImaging2006}.

\section{Analysis} \label{sec:analysis}

\begin{figure*}
    \centering
    \epsscale{1}
    \plotone{contrast_curves}
    \caption{The contrast curves for the best performing algorithm from each epoch. The solid lines are the Gaussian 5$\sigma$ contrast curves and the dashed lines are the Student-t corrected curves. The expected upper limit for a dynamically stable orbit of \qty{1.5}{\au} is plotted as a vertical dashed line. The annular PCA curve is cut off because the innermost annulus has a contrast value greater than 1, which means even a 100 S/N companion would not be detectable with 5$\sigma$ significance.}
    \label{fig:contrast}
\end{figure*}

\begin{figure*}
    \centering
    \epsscale{1}
    \plotone{mass_curves}
    \caption{Mass sensitivity curves derived from the 2020-11-21 epoch, which has the best contrast. The solid lines are the Gaussian 5$\sigma$ detection limits and the dashed lines are the Student-t corrected limits. The two ages represent the ages of the two potential formation pathways, one of which is the system age (\qty{226}{\mega\year}), the other is the WD cooling age of Sirius B (\qty{125}{\mega\year}). The limits are calculated assuming an Lp apparant magnitude for Sirius B of \num{9.01} and converting contrast into absolute magnitudes. The absolute magnitudes are converted to masses using the ATMO2020 isochrone grid with non-equilibrium chemistry and weak convective mixing. The lower mass limit of the ATMO2020 grid is plotted with a horizontal dashed line and the expected upper limit for dynamically stable orbits of \qty{1.5}{\au} is plotted with a vertical dashed line.}
    \label{fig:mass}
\end{figure*}

\subsection{Pre-processing}

The raw images from NIRC2 required pre-processing before analyzing them for companions. For each epoch, we applied a flat correction using calibration frames captured during observing. We also removed bad pixels using a combination of \texttt{L.A.Cosmic} \citep{vandokkumCosmicRayRejectionLaplacian2001a} and an adaptive sigma-clipping algorithm. We removed sky background using a high-pass median filter with a box size of 31 pixels. For both the November epochs we tried exploiting the large focal plane drifts by dithering between two positions in order to simplify background subtraction, but this ended up performing worse than the high-pass filter. At this point frames were manually selected to remove bad frames, especially those with diffraction spikes from Sirius A within a few hundred pixels, like in \cref{fig:spike}. Then, each good frame was co-registered to sub-pixel accuracy using the algorithm presented in \citet{guizar-sicairosEfficientSubpixelImage2008}, followed by fitting each frame with a Gaussian PSF to further increase centroid accuracy.

The co-registered frames were shifted to the center of the FOV and cropped to the inner 200 pixels. With a pixel scale of \qty{10}{\milliarcsecond} the crop corresponds to a maximum separation of \ang{;;1} or a projected separation of \qty{2.7}{\au}. All the frames were stacked into data cubes for each epoch. We also measure the parallactic angle of each frame, including corrections for distortion effects following \citet{yeldaImprovingGalacticCenter2010}. For each epoch, we measure the full-width at half-maximum (FWHM) of the stellar PSF for use in post-processing by fitting a bivariate Gaussian model to the median frame from each data cube (\cref{fig:psf}). All of the pre-processing code is available in Jupyter notebooks in a GitHub repository (\cref{sec:data}).

\subsection{Post-processing}

\begin{figure*}
    \centering
    \epsscale{0.95}
    \plotone{residuals}
    \caption{The flat residuals of each epoch after PSF subtraction, derotating, and collapsing. The inner full-width at half-maximum (FWHM) is masked out for each frame.}
    \label{fig:residuals}
\end{figure*}

\begin{figure*}
    \centering
    \epsscale{0.95}
    \plotone{sig}
    \caption{The \textit{significance} maps for each epoch accounting for small-sample statistics \citep{mawet_fundamental_2014}. Typically a critical value for detection is 5. The inner full-width at half-maximum (FWHM) is masked out for each map.}
    \label{fig:sig}
\end{figure*}

\begin{figure*}
    \centering
    \epsscale{0.95}
    \plotone{stim}
    \caption{The STIM maps for each epoch calculated from each residual cube. The STIM probability has a typical cutoff threshold of 0.5 for significant detections. The inner full-width at half-maximum (FWHM) is masked out for each map.}
    \label{fig:stim}
\end{figure*}

By taking data with the field rotator disabled (ADI), the point-spread function (PSF) will not appear to rotate while any potential companion will appear to rotate. This reduces the probability of removing companion signal when we subtract the stellar PSF model. After subtraction, the frames are derotated by their parallactic angle and combined with a weighted sum \citep{bottomNoiseweightedAngularDifferential2017a}, which reduces the pixel-to-pixel noise as the number of frames in the data cube increases.

For this analysis, we used four ADI algorithms for modeling and subtracting the stellar PSF: median subtraction \citep{maroisAngularDifferentialImaging2006}, principal component analysis \citep[PCA, also referred to as KLIP;][]{soummerDetectionCharacterizationExoplanets2012a}, non-negative matrix factorization \citep[NMF;][]{renNonnegativeMatrixFactorization2018a}, and fixed-point greedy disk subtraction \citep[GreeDS;][]{papairetReferencelessAlgorithmCircumstellar2019a,pairetMAYONNAISEMorphologicalComponents2020}. The median subtraction and PCA methods were also applied in an annular method, where we modeled the PSF in annuli of increasing separation, discarding frames that have not rotated at least 0.5 FWHM \citep{maroisAngularDifferentialImaging2006}. These algorithms are implemented in the open-source \texttt{ADI.jl} Julia package \citep{lucasADIJlJulia2020}.

We used three metrics available in \texttt{ADI.jl} to determine the performance of each algorithm, the signal-to-noise ratio (S/N) significance map, the standardized trajectory intensity mean map \citep[STIM map;][]{pairetSTIMMapDetection2019}, and the contrast curve. The significance and STIM maps assign a likelihood to each pixel for the presence of a companion using different assumptions of the residual statistics. The contrast curve determines the sensitivity of a 5$\sigma$ statistical detection through repeated injection and retrieval of planetary signal as processed by one of the ADI algorithms above. We calculate both the Gaussian contrast and the Student-t corrected contrast, which accounts for the small-sample statistics in each annulus \citep{mawetFundamentalLimitationsHigh2014}. The collapsed residual frames along with the above metrics for each algorithm for each epoch are in section~\ref{sec:adi-results}.

A common problem when using subspace-driven post-processing algorithms like PCA, NMF, or GreeDS is choosing the size of the subspace (i.e., the number of components). For PCA, NMF, and GreeDS algorithms, we increased the number of components from 1 to 10, creating a residual cube for each iteration. We chose 10 for the maximum number of components because we saw a dramatic decline in contrast sensitivity after the first few components (\cref{fig:pca-contrast-curves}). In our analysis we employed the STIM largest intensity mask map \citep[SLIM map;][]{pairetSignalProcessingMethods2020} as an ensemble statistic. The SLIM map calculates the average STIM map from many residual cubes along with the average mask of the $N$ most intense pixels in each STIM map. A real companion ought to be present in many different residual cubes from the same dataset, so this ensemble approach can give us a probability map without predetermining the number of components. The collapsed residual frames, average STIM map, SLIM map, and contrast curves for each epoch for each of the above algorithms are in section~\ref{sec:adi-results}. All of the code used for this analysis is available in a GitHub respository (\cref{sec:data}).


\section{Results} \label{sec:results}

We determined the best-performing algorithms for each epoch using the contrast curves described in \cref{sec:analysis}. For the first two epochs, full-frame median subtraction had the best contrast at almost all separations. For the last epoch annular PCA subtraction with 2 principal components and a rotation threshold of 0.5 FWHM produced the best contrast at close separations (\qtyrange{0.2}{0.4}{\arcsecond}) and had similar performance to other algorithms beyond \ang{;;0.4}. The innermost annulus from this algorithm has greater than 1 contrast which implies a 100 S/N companion would not be detected with 5$\sigma$ significance. The contrast for this innermost annulus is therefore not plotted. The collapsed residual frames from each epoch are shown in \cref{fig:residuals}, along with the Gaussian significance maps (\cref{fig:sig}) and STIM maps (\cref{fig:stim}).

The reduced images do not show consistent or significant evidence for a substellar companion. The STIM probability maps for the 2020-11-21 and 2020-11-28 epochs suggest evidence for some blobs $\sim$\qty{0.3}{\au} (\ang{;;0.13}; 1.6 FWHM) from the center. The February epoch also shows a blob at a similar separation in the reduced image which does not appear in the STIM map. The lack of statistical evidence in the February epoch and the significance maps as well as the proximity to the central star both reduce the probability of these blobs being true companions. Nonetheless, we estimated astrometry for blobs from each epoch (\cref{tbl:astrometry}) and tried fitting Keplerian orbits using the ``Orbits for the Impatient'' algorithm \citep[OFTI;][]{bluntOrbitsImpatientBayesian2017a}. We generated $10^4$ orbits, none of which contained the points from each epoch (section~\ref{sec:orbits}). We take this as direct evidence against the blobs being substellar companions of any kind.


\subsection{Companion Mass Detection Limits}

The contrast maps from the best performing algorithm for each reduction are shown in \cref{fig:contrast}. We determine the limiting sensitivities in terms of the planetary mass by first calculating the contrast-limited magnitude using an Lp-band magnitude for Sirius B of 9.1 (adapted from \citealp{bidaudADONISHighContrast2008a}). Then we used \qty{225}{\mega\year} for the system age to interpolate the planetary mass using the AMES-Cond evolutionary grid and atmosphere models . The high precision of the Sirius system's age reduces uncertainty when interpolating planetary mass from the evolutionary grids (see \cref{sec:intro}). We also tested using the newer SONORA grid , but the differences compared to AMES-Cond were less than the mean uncertainty from age. The best performing epoch was on the night of 2020-11-21, which managed to reach an exceptional sensitivity of \qty{3.5}{\jupitermass} at \qty{0.25}{\au} (\ang{;;0.09}) in the speckle limited regime and ultimately \qty{0.6}{\jupitermass} at \qty{1.5}{\au} (\ang{;;0.38}) in the background limited regime.

\section{Conclusions} \label{sec:conclusion}

In closing, the Sirius system is one of the most well studied in history, with Sirius B being the target of companion searches from the visible to the IR. While it is highly unlikely a first-generation planet survived post-MS evolution, imaging efforts have gradually increased the sensitivity to second-generation planets. In this work, we present high-contrast images of Sirius B in the near-IR. Our sensitivity limits are the best that have been reached for Sirius B so far, reaching \qty{0.6}{\jupitermass} at \qty{1.5}{\au}, the outer limit for dynamically stable orbits. Our observations also show how the high precision of the parameters of the Sirius system directly benefits the sensitivity to planets. Particularly, the low age uncertainty of Sirius B keeps our mass uncertainty below \qty{0.1}{\jupitermass}. Despite the high sensitivity of this study, we found no appreciable evidence for a companion around Sirius B, consistent with previous results. This non-detection furthers evidence against a second-generation planet migrating or forming within the Sirius system \citep{viganHighcontrastImagingSirius2015}.

For future work, we consider two avenues for followup. As mentioned in \cref{sec:results}, optical observations using an instrument like Subaru/VAMPIRES . Optical observations can look for disks in the Rayleigh-scattering regime, which is not readily probed by IR observations. Furthermore, the polarimetric differential observation modes of both instruments allow for nearly diffraction-limited imaging, down to 10s of milliarcseconds.

Followup in the IR from space-based telescopes would increase the sensitivity of the observations due to the absence of background emission from Earth's atmosphere. For example, using JWST/NIRCAM in long-wavelength imaging mode has a limiting magnitude of $\sim$25 in the F480M filter, which would easily reach the model mass sensitivity limits at \textgreater\ang{;;0.3} without saturation . The pixel scale (\qty{0.06}{\arcsecond\per\pixel}) and PSF size ($\sim$\ang{;;0.3}) would still probe down to \qty{0.8}{\au}. Observations would be complicated by the scattered light from Sirius A, which could affect the fine-pointing control or worsen the background sensitivity. In the end, our ground-based observations are already quite close to the atmospheric model grid limits, raising the concern of diminishing returns. Planetary atmosphere models must improve, before fully reaping the benefits of the additional sensitivity of space-based observations.

\subsection{Data and Code Availability} \label{sec:data}
All of the code used for pre-processing data, reducing data, and generating the figures is available under an open-source license in a GitHub repository\footnote{\href{https://github.com/mileslucas/sirius-b}{https://github.com/mileslucas/sirius-b}}. This code includes all of the scripts for generating each figure in this manuscript. The pre-processed data cubes and parallactic angles are available on Zenodo under an open-source license\footnote{\dataset[10.5281/zenodo.5115225]{\doi{10.5281/zenodo.5115225}}}. We hope that this improves the reproducibility of the work as well as providing data for future investigations. Please reach out to the corresponding author for further inquiries regarding data and code availability.

\begin{acknowledgements}
The data presented herein were obtained at the W. M. Keck Observatory, which is operated as a scientific partnership among the California Institute of Technology, the University of California, and the National Aeronautics and Space Administration. The Observatory was made possible by the generous financial support of the W. M. Keck Foundation. The authors wish to recognize and acknowledge the very significant cultural role and reverence that the summit of Maunakea has always had within the indigenous Hawaiian community. We are most fortunate to have the opportunity to conduct observations from this mountain.
\end{acknowledgements}


\facility{Keck:II (NIRC2)}

\software{
ADI.jl \citep{lucasADIJlJulia2020},
astropy \citep{cocollaborationAstropyCommunityPython2013,astropycollaborationAstropyProjectBuilding2018},
Julia \citep{bezansonJuliaFreshApproach2017},
numpy \citep{harrisArrayProgrammingNumPy2020},
scikit-image \citep{waltScikitimageImageProcessing2014},
}

\bibliography{references}{}
\bibliographystyle{aasjournal}

\appendix

\section{ADI Processing Results} \label{sec:adi-results}

Results from processing each epoch of Sirius B data with various ADI algorithms. The top-left image is the subtracted, derotated, and collapsed image. The top-middle image is the Gaussian S/N map, the bottom-middle image is the Student-t S/N (significance) map, which typically requires a value of 5 to show a significant detection, and the bottom-left image is the STIM probability map, which typically requires a probability of 0.5 for significant detection. The right graph shows the 5$\sigma$ contrast, both the Gaussian and Student-t corrected curves.    


\begin{figure}[h!]
    \centering
    \epsscale{0.9}
    \plotone{reports/2020feb04_median}
    \caption{2020-02-04 median}
\end{figure}

\begin{figure}[h!]
    \centering
    \epsscale{0.9}
    \plotone{reports/2020feb04_pca-2}
    \caption{2020-02-04 PCA(2)}
\end{figure}

\begin{figure}[h!]
    \centering
    \epsscale{0.9}
    \plotone{reports/2020feb04_nmf-2}
    \caption{2020-02-04 NMF(2))}
\end{figure}

\begin{figure}[h!]
    \centering
    \epsscale{0.9}
    \plotone{reports/2020feb04_greeds-2}
    \caption{2020-02-04 GreeDS(2)}
\end{figure}

\begin{figure}[h!]
    \centering
    \epsscale{0.9}
    \plotone{reports/2020feb04_annular_median}
    \caption{2020-02-04 annular median}
\end{figure}

\begin{figure}[h!]
    \centering
    \epsscale{0.9}
    \plotone{reports/2020feb04_annular_pca-2}
    \caption{2020-02-04 annular PCA(2))}
\end{figure}

\begin{figure}[h!]
    \centering
    \epsscale{0.9}
    \plotone{reports/2020feb04_annular_nmf-2}
    \caption{2020-02-04 annular NMF(2)}
\end{figure}

\begin{figure}[h!]
    \centering
    \epsscale{0.9}
    \plotone{2020feb04_contrast_curves}
    \caption{5$\sigma$ contrast curves from various ADI algorithms for the 2020-02-04 epoch. Both the Gaussian (solid lines) and Student-t corrected (dashed lines) contrast curves are shown.}
\end{figure}

%%%%%%%%%%%%%%%%%

\begin{figure}[h!]
    \centering
    \epsscale{0.9}
    \plotone{reports/2020nov21_median}
    \caption{2020-11-21 median subtraction.}
\end{figure}

\begin{figure}[h!]
    \centering
    \epsscale{0.9}
    \plotone{reports/2020nov21_pca-2}
    \caption{2020-11-21 PCA subtraction with 2 components.}
\end{figure}

\begin{figure}[h!]
    \centering
    \epsscale{0.9}
    \plotone{reports/2020nov21_nmf-2}
    \caption{2020-11-21 NMF subtraction with 2 components.}
\end{figure}

\begin{figure}[h!]
    \centering
    \epsscale{0.9}
    \plotone{reports/2020nov21_greeds-2}    
    \caption{2020-11-21 GreeDS with 2 components.}
\end{figure}

\begin{figure}[h!]
    \centering
    \epsscale{0.9}
    \plotone{reports/2020nov21_annular_median}
    \caption{2020-11-21 annular median subtraction with a rotation threshold of 0.5.}
\end{figure}

\begin{figure}[h!]
    \centering
    \epsscale{0.9}
    \plotone{reports/2020nov21_annular_pca-2}
    \caption{2020-11-21 annular PCA subtraction with 2 components and a rotation threshold ofnov21 0.5.}
\end{figure}

\begin{figure}[h!]
    \centering
    \epsscale{0.9}
    \plotone{reports/2020nov21_annular_nmf-2}
    \caption{2020-11-21 annular NMF subtraction with 2 components and a rotation threshold of 0.5.}
\end{figure}

\begin{figure}[h!]
    \centering
    \epsscale{0.9}
    \plotone{2020nov21_contrast_curves}
    \caption{5$\sigma$ contrast curves from various ADI algorithms for the 2020-11-21 epoch. Both the Gaussian (solid lines) and Student-t corrected (dashed lines) contrast curves are shown.}
\end{figure}

%%%%%%%%%%%%%%%%%

\begin{figure}[h!]
    \centering
    \epsscale{0.9}
    \plotone{reports/2020nov28_median}
    \caption{2020-11-28 median subtraction.}
\end{figure}

\begin{figure}[h!]
    \centering
    \epsscale{0.9}
    \plotone{reports/2020nov28_pca-2}
    \caption{2020-11-28 PCA subtraction with 2 components.}
\end{figure}

\begin{figure}[h!]
    \centering
    \epsscale{0.9}
    \plotone{reports/2020nov28_nmf-2}
    \caption{2020-11-28 NMF subtraction with 2 components.}
\end{figure}

\begin{figure}[h!]
    \centering
    \epsscale{0.9}
    \plotone{reports/2020nov28_greeds-2}
    \caption{2020-11-28 GreeDS with 2 components.}
\end{figure}

\begin{figure}[h!]
    \centering
    \epsscale{0.9}
    \plotone{reports/2020nov28_annular_median}
    \caption{2020-11-28 annular median subtraction with a rotation threshold of 0.5.}
\end{figure}

\begin{figure}[h!]
    \centering
    \epsscale{0.9}
    \plotone{reports/2020nov28_annular_pca-2}
    \caption{2020-11-28 annular PCA subtraction with 2 components and a rotation threshold ofnov28 0.5.}
\end{figure}

\begin{figure}[h!]
    \centering
    \epsscale{0.9}
    \plotone{reports/2020nov28_annular_nmf-2}
    \caption{2020-11-28 annular NMF subtraction with 2 components and a rotation threshold of 0.5.}
\end{figure}

\begin{figure}[h!]
    \centering
    \epsscale{0.9}
    \plotone{2020nov28_contrast_curves}
    \caption{5$\sigma$ contrast curves from various ADI algorithms for the 2020-11-28 epoch. Both the Gaussian (solid lines) and Student-t corrected (dashed lines) contrast curves are shown.}
\end{figure}


\clearpage
\section{Provisional Orbit Fitting} \label{sec:orbits}

We found multiple interesting blobs in the reduced data that were not statistically significant. Nonetheless, we tried fitting Keplerian orbits using OFTI to determine the feasibility of the blobs being real companions. We began by estimating the astrometry of the blobs by eye in reduced data (\cref{tbl:astrometry}, \cref{fig:prov-orbit}). We tried simulating $10^4$ orbits via rejection sampling with OFTI but no generated orbit was able to constrain all three points. Overall we determine these blobs are not real companions and are most likely systematic noise in the stellar PSF.

\begin{deluxetable}{ccc}[h!]
    \tablecaption{Provisional astrometry for a blob of interest from each epoch. The separation and offset are in relation to Sirius B. The uncertainties are derived from the FWHM of the PSF from each epoch.
    \label{tbl:astrometry}}
    \tablehead{
        \colhead{Date observed} &
        \colhead{offset (\unit{\milliarcsecond})} &
        \colhead{PA (\unit{\degree})}
    }
    \startdata
    2020-02-04 & $123\pm 40$ & $-128\pm 20$  \\
    2020-11-21 & $119\pm 38$ & $68\pm 18$  \\
    2020-11-28 & $132\pm 41$ & $37\pm 21$  \\
    \enddata
\end{deluxetable}

\begin{figure}[h!]
    \centering
    \figurenum{14}
    \plotone{orbit_frames}
    \caption{Provisional astrometry (white circles) displayed on collapsed and derotated residuals from each epoch. Each frame was cropped to the inner $\sim$\qty{0.7}{\au} (\ang{;;0.25}) and the inner FWHM has been masked out. The width of the circles represents the uncertainty.}
    \label{fig:prov-orbit}
\end{figure}

\end{document}
