\documentclass[twocolumn]{aastex631}
\usepackage{hyperref}
\let\tablenum\relax
\usepackage{siunitx}
\usepackage{makecell}


\DeclareSIUnit\parsec{pc}
\DeclareSIUnit\au{AU}
\DeclareSIUnit\jupitermass{M_J}
\DeclareSIUnit\milliarcsecond{mas}
\DeclareSIUnit\solarmass{M_\odot}
\DeclareSIUnit\year{yr}
\sisetup{
    list-units=single,
    separate-uncertainty
}

%% Define new commands here
\newcommand\latex{La\TeX}

\graphicspath{{./}{figures/}}
\shortauthors{Lucas et al.}

\begin{document}

\title{An Imaging Search for Circumstellar Companions of Sirius B}

\shorttitle{An Imaging Search for Circumstellar Companions of Sirius B}
\shortauthors{Lucas et al.}

\correspondingauthor{Miles Lucas}
\email{mdlucas@hawaii.edu}

\author[0000-0001-6341-310X]{Miles Lucas}
\affiliation{Institute for Astronomy, University of Hawai'i, USA}

\author[0000-0003-1341-5531]{Michael Bottom}
\affiliation{Institute for Astronomy, University of Hawai'i, USA}

\author[0000-0003-4769-1665]{Garreth Ruane}
\affiliation{Jet Propulsion Laboratory, California Institute of Technology, USA}

\author[0000-0002-0696-1780]{Sam Ragland}
\affiliation{W.M. Keck Observatory, USA}


\begin{abstract}
We present deep imaging of Sirius B, the closest and brightest white dwarf. We use Keck/NIRC2 in Lp band (\qty{3.776}{\micro\meter}) across 3 epochs in 2020 using the technique of angular differential imaging. We reach sub-Jupiter sensitivities and sub-AU separations, reaching $\sim$\qty{3.5}{\jupitermass} at $\sim$\qty{0.25}{\au} down to a sensitivity of $\sim$\qty{0.6}{\jupitermass} at $\sim$\qty{1.5}{\au}. We do not detect any companions around Sirius B, consistent with previous studies. Our observations are ultimately limited by the sky background more than contrast or separation, which is promising for future imaging studies of nearby faint targets.
\end{abstract}

\section{Introduction} \label{sec:intro}

High-contrast imaging (HCI) is a powerful technique for discovering and characterizing exoplanets. Being able to probe the architecture, formation, and atmospheres of planets directly is necessary for advancing substellar companion formation and evolution theory. The process required to image a planet is daunting, however. The typical astrophysical flux ratios for a Sun-Jupiter analog in the near-infrared (NIR) are $\sim$$10^{-8}$, and for a Sun-Earth system are $\sim$$10^{-10}$ \citep{traub_direct_2010}. The infrared is where blackbody emission from exoplanets peaks, while simultaneously being well into the Rayleigh-Jeans limit of the star, decreasing contrast compared to the visible. In addition, the close angular separations of planets makes it difficult to detect them over the diffraction pattern of their host and other noise sources.

Typical targets for imaging are nearby young stars; the close proximity means the same angular separation probes a closer projected separation, allowing closer investigation, and younger exoplanets are hotter and therefore brighter, reducing the contrast. Another way to reduce contrast would be to make the star fainter, which is precisely what happens in the latest stages of stellar evolution during the white dwarf cooling sequence. When intermediate mass stars (\qtyrange{1}{8}{\solarmass}) will eventually exit the main-sequence (MS) and expand up to $\sim$100s of stellar radii during the red giant branch (RGB) phase, potentially engulfing planets within the star's roche limit. Then, during the asymptotic giant branch (AGB) phase, mass loss will adiabatically expand any orbits. Once the star becomes a white dwarf it begins cooling, which reduces its luminosity by 3 to 4 orders of magnitude compared to its MS progenitor. The faintness broad spectral lines, and expanded orbits make white dwarfs exceptionally challenging for the transit photometry and radial velocity methods. However, the faintness reduces the contrast of planetary companions and the expanded separation greatly benefit imaging.

White dwarfs do present challenges, and there is limited knowledge of planetary systems around evolved stars. \citet{burleigh_imaging_2002,veras_post-main-sequence_2016} suggests exoplanets on initially wide ($>$\qty{5}{\au}) orbits around intermediate mass stars will survive expansion during the RGB phase. Close-in planets that escape the Roche limit of the expanding red giant can still be shredded by tidal forces, although the tidal forces on planets that readily escape engulfment are negligible \citep{nordhaus_orbits_2013}. During the AGB phase, stellar mass loss will adiabatically expand the orbit of the planet by a maximum factor of $M_{*,\mathrm{MS}}/M_{*,\mathrm{WD}}$ \citep{jeans_cosmogonic_1924}. In addition to \textit{first-generation} planets there are potential methods for \textit{second-generation} planets to enter the system after the violent RGB and AGB phases \citep{perets_second_2010}.

The first search for substellar companions around white dwarfs was conducted by \citet{probst_infrared_1983} by searching for infrared (IR) excess in their spectral energy distributions (SED) using broadband photometry. They found no companions around the $\sim$100 white dwarfs they studied. The same method was applied by \citet{zuckerman_excess_1987} who found excess IR emission around white dwarf G29-38. Eventually this excess was determined to be from a dust disk \citep{telesco_observations_1990} which is indicative of the planetary formation history. The current interpretation associates the dust with accretion onto the white dwarf, polluting the stellar atmosphere \citep{koester_metals_1997}.

To date, there has been no direct images of an exoplanet around a white dwarf, although a planetary-mass brown dwarf with a mass of \qty{7}{\jupitermass} on a wide \qty{2500}{\au} orbit was imaged around a single white dwarf (WD 0806-661B) using Spitzer \citep{luhman_discovery_2011}. The ``Degenerate Objects around Degenerate Objects'' survey \citep[DODO;][]{hogan_dodo_2009} observed 29 white dwarfs with Gemini/NIRI and VLT/ISAAC, reaching an average upper limit around $\sim$\qty{8}{\jupitermass} beyond \qty{35}{\au}. The young white dwarf GD 50 was observed using the extreme AO instrument SPHERE at the VLT \citep{xu_extreme-ao_2015}, reaching sensitivity limits of \qty{4}{\jupitermass} at \qty{6.2}{\au}.

A particularly fascinating white dwarf is Sirius B, the closest and brightest white dwarf. The Sirius system is the 7th closest to the sun at \qty{2.6}{\parsec}, consisting of Sirius A, a -1.35 magnitude A1Vm star known for being the brightest, and Sirius B, a DA2 white dwarf with a \qty{50}{\year} orbit \citep{bond_sirius_2017,gaia_collaboration_gaia_2018}. As mentioned previously, the proximity and faintness of Sirius B (compared to a MS star) make it compelling for imaging, and additionally it is a young system ($\sim$\qty{225}{\mega\year}), further reducing the predicted contrast necessary to image a planetary companion.

The orbit of Sirius B was first studied by \citet{bessel_variations_1844}, who recognized wobbles in the proper motion of Sirius A caused by a ``dark satellite''. This dark satellite was visually confirmed in \citet{bond_companion_1862} as Sirius B. \citet{adams_spectrum_1915} took the first spectral measurements of Sirius B and found it to be similar to a MS early A-type star, despite its faintness, which we now know to be typical of white dwarf spectra. 

Initial astrometric perturbations suggested a 50 year orbital period \citep{auwers_orbit_1864}, with the first orbital solution including dynamic masses published by \citet{van_den_bos_orbit_1960}. \citet{gatewood_study_1978} improved upon the \citeauthor{van_den_bos_orbit_1960} orbital solution with over 300 more images, finding nearly the same dynamical masses for the binary. \citet{bond_sirius_2017} greatly refined the orbit using a compilation of historical data and Hubble Space Telescope (HST) data, which gave dynamical masses of \qty{2.063+-0.023}{\solarmass} and \qty{1.018+-0.011}{\solarmass} for A and B, respectively.

The precise dynamical masses are key for finding the age of Sirius; \citet{bond_sirius_2017} report a cooling age for Sirius B of \qty{126}{\mega\year} using isochrones. Studies of the initial-final mass relation (IFMR) of white dwarfs \citep{cummings_two_2016} lead to estimates of the progenitor mass of Sirius B to be between \qtyrange{5}{5.6}{\solarmass}, which, when combined with stellar evolution codes, yield total ages between \qtyrange{226}{228}{\mega\year} with an uncertainty of about $\pm$\qty{10}{\mega\year} \citep{bond_sirius_2017}. The precision of the system age is useful for constraining the mass of potential companions.

A companion around Sirius B would be affected by the orbit of Sirius A, and this constrained three-body system has been studied numerically \citep{holman_long-term_1999}. \citet{bond_sirius_2017} calculate the longest period stable companion around Sirius B is \qty{1.79}{\year}, which corresponds to a \qty{1.5}{\au} circular orbit.

The first modern imaging study searching for companions around Sirius B was \citet{schroeder_search_2000} who used the HST wide-field planetary camera (WFPC) at \qty{1}{\micro\meter}. Around the same time \citet{kuchner_search_2000} searched in a narrower field of view (FOV) with HST/NICMOS at \qty{1}{\micro\meter}. These studies combined had a sensitivity down to $\sim$\qty{10}{\jupitermass} at \qty{5.3}{\au} (\ang{;;2}). \citet{bonnet-bidaud_adonis_2008} used the ground-based ESO/ADONIS instrument in J, H, and Ks-band and reached a sensitivty of $\sim$\qty{30}{\jupitermass} at \qty{7.9}{\au} (\ang{;;3}). \citet{skemer_sirius_2011} used mid-IR (up to \qty{10}{\micro\meter}) observations from Gemini/T-ReCs which ruled out evidence for any infrared excess around Sirius B. \citet{thalmann_piercing_2011} used Subaru/IRCS at \qty{4.05}{\micro\meter} reaching detection sensitivities of \qtyrange{6}{12}{\jupitermass} at \ang{;;1}. Recently, \citet{pathak_high_2021} took simultaneous mid-IR observations (\qty{10}{\micro\meter}) at VLT/VISIR of Sirius A (through a coronagraph) and B. Because of the simultaneous observation, their contrast depended on which region of the FOV was tested. Their average sensitivity is $\sim$\qty{2.5}{\jupitermass} at \qty{1}{\au}, and their best sensitivity (from the ``inner'' region) is $\sim$\qty{1.5}{\jupitermass} at \qty{1}{\au}.

In this work we report direct images of Sirius B with Keck/NIRC2. \autoref{sec:obs} describes our target and observing strategy, \autoref{sec:analysis} describes the processing and analysis techniques used, \autoref{sec:results} describes our results and we conclude with \autoref{sec:conclusion}.

\section{Observations} \label{sec:obs}

Despite Sirius B being the brightest white dwarf in the sky, it is still 10 magnitudes fainter than Sirius A, making it a technically challenging target, especially on ground-based telescopes. We targeted Sirius B directly using Keck/NIRC2 in Lp-band (\qty{3.776}{\micro\meter}) across three epochs in 2020 (\autoref{tbl:obs}). Our first attempt to observe Sirius B failed due to the strong scattered light from Sirius A. The adaptive optics (AO) calibration failed when the scattered light from Sirius A would sweep into the FOV of the wavefront sensor (WFS). Similarly, trying to deploy the focal-plane vortex coronagraph \citep{serabyn_w_2017} failed when the coronagraphic pointing control algorithm, QACITS \citep{huby_w_2017}, performed erratically in the presence of the scattered light. \citet[\S2]{vigan_high-contrast_2015} reported similar issues in their attempts to image Sirius B. In order to overcome these obstacles, we decided to try using Sirius A as the AO guide star and off-axis guiding to Sirius B.

The Keck AO system \citep{wizinowich_performance_2000} was not designed to accommodate stars as bright as Sirius A, so we needed to attenuate the flux greatly to avoid saturating the WFS. We experimented with different filters in front of the WFS, settling on a narrow laser-line filter which attenuated Sirius A by $\sim$6 magnitudes. While still bright (appearing like a $\sim$5 magnitude star on the WFS), this was enough attenuation to close the AO loop. From here, we slewed off-axis using the separations and position angles calculated in \autoref{tbl:obs} from the orbital parameters in \cite{bond_sirius_2017}. In this mode, we noticed higher than usual drift in the focal plane, requiring manually recentering the target every 5 or 10 minutes. We tried deploying the vortex coronagraph along with QACITS, but again the scattered light from Sirius A made QACITS unstable, so we decided to forego any coronagraphy for the remaining observations.

During each observation, we took calibration frames in the form of dark frames and sky flat frames. We also disabled the field rotator, which caused the FOV to rotate throughout the night for angular differential imaging \citep[ADI;][]{marois_angular_2006}.

\begin{figure}
    \centering
    \epsscale{1.1}
    \plotone{spike}
    \caption{Calibrated science frame of Sirius b from 2020-02-04 epoch showing the strong scattered light effects from Sirius a.}
    \label{fig:spike}
\end{figure}


\begin{deluxetable*}{ccccccccc}
    \tabletypesize{\small}
    \tablecaption{Observing parameters for the three epochs of data. All observations were carried out using the NIRC2 Lp-band filter. Observation time is based on the frames that were selected for processing.
    \label{tbl:obs}}
    \tablehead{
        \colhead{\makecell{Date\\observed}} &
        \colhead{\makecell{Sirius B\\offset} (\unit{\arcsecond})} &
        \colhead{\makecell{Sirius B\\PA} (\unit{\degree})} &
        \colhead{\makecell{Obs.\\time} (hr)} & 
        \colhead{\makecell{FOV\\rotation} (\unit{\degree})} &
        \colhead{FWHM (\unit{\milliarcsecond})} &
        \colhead{Seeing (\unit{\arcsecond})} & 
        \colhead{Temp (\unit{\celsius})} & 
        \colhead{PWV (\unit{\milli\meter})}
    }
    \startdata
    2020-02-04 & 11.20 & 67.90 & 1.44 & 60.1 & 79.9 & &  &  \\
    2020-11-21 & 11.27 & 66.42 & 2.91 & 91.4 & 76.4 & &  &  \\
    2020-11-28 & 11.27 & 66.38 & 2.44 & 80.4 & 82.2 & &  &  \\
    \enddata
\end{deluxetable*}

\section{Analysis} \label{sec:analysis}

For each epoch we applied a flat correction using calibration frames captured during observing. We also removed bad pixels using a combination of L.A.Cosmic \citep{van_dokkum_cosmic-ray_2001} and an adaptive sigma-clipping algorithm. We removed sky background using a high-pass median filter with a box size of 31 pixels. For both the November epochs we tried exploiting the large focal plane drifts by dithering between two positions in order to simplify background subtraction, but this ended up performing worse than the high-pass filter. At this point frames were manually selected to remove bad frames, especially those with diffraction spikes from Sirius A within a few hundred pixels, like in \autoref{fig:spike}. Then, each good frame was co-registered to sub-pixel accuracy using the algorithm presented in \citet{guizar-sicairos_efficient_2008}, followed by fitting each frame with a Gaussian PSF to reach floating-point centroid accuracy.

The co-registered frames are then shifted to the center of the FOV. Lastly the frames were cropped to the inner 200 pixels and stacked into data cubes for each epoch. With a pixel scale of \qty{10}{\milliarcsecond} the crop corresponds to a maximum separation of \ang{;;1} or a projected separation of \qty{2.7}{\au}. We also measure the parallactic angle of each frame, including corrections for distortion effects following \citet{yelda_improving_2010}. For each epoch, we measure the full-width at half-maximum (FWHM) of the stellar PSF for use in post-processing by fitting a bivariate Gaussian model to the median frame from each data cube (\autoref{fig:psf}). All of the pre-processing code is available in Jupyter notebooks at the following GitHub repository\footnote{\href{https://github.com/mileslucas/sirius-b}{https://github.com/mileslucas/sirius-b}} and the pre-processed data cubes and parallactic angles are available on Zenodo\footnote{\citealt{lucas_nirc2_2021}}.

\begin{figure}
    \centering
    \epsscale{1.1}
    \plotone{psf}
    \caption{The median frame from the 2020-11-21 epoch showing the instrumental PSF. The inner core has a FWHM of $\sim$\qty{76}{\milliarcsecond}. The speckle pattern is shown in the blobs surrounding the first ring, with roughly 6-way radial symmetry corresponding to the hexagonal shape of the primary mirror.}
    \label{fig:psf}
\end{figure}

By taking data with the field rotator disabled (ADI), the point-spread function (PSF) will not appear to rotate while any potential companion will appear to rotate. This reduces the probability of subtracting companion signal when we subtract the stellar PSF model. After subtraction, the frames are derotated by their parallactic angle and combined with a weighted sum \citep{bottom_noise-weighted_2017}, which reduces the pixel-to-pixel noise as the number of frames in the data cube increases.

For this analysis we used four ADI algorithms for modeling and subtracting the stellar PSF: median subtraction \citep{marois_angular_2006}, principal component analysis \citep[PCA, also referred to as KLIP;][]{soummer_detection_2012}, non-negative matrix factorization \citep[NMF;][]{ren_non-negative_2018}, and fixed-point greedy disk subtraction \citep[GreeDS;][]{pairet_reference-less_2019,pairet_mayonnaise_2020}. The median subtraction and PCA methods were also applied in an annular method, where we modeled the PSF in annuli of increasing separation frame-by-frame, discarding frames which have not rotated at least 0.5 FWHM \citep{marois_angular_2006}.

We used three metrics for determining the performance of each algorithm, the signal-to-noise ratio (S/N) significance map, the standardized trajectory intensity mean map \citep[STIM map;][]{pairet_stim_2019}, and the contrast curve. The significance and STIM maps assign a likelihood to each pixel for the presence of a companion using different assumptions of the residual statistics. The contrast curve determines the sensitivity of a 5$\sigma$ statistical detection through repeated injection and retrieval of planetary signal as processed by one of the ADI algorithms above. We calculate both the Gaussian contrast and the Student-t corrected contrast, which accounts for the small-sample statistics in each annulus \citep{mawet_fundamental_2014}. The collapsed residual frames along with the above metrics for each algorithm for each epoch are in \autoref{sec:adi-results}.

A common problem when using subspace-driven post-processing algorithms like PCA, NMF, or GreeDS is choosing the size of the subspace (i.e., the number of components). For PCA, NMF, and GreeDS algorithms, we created residual cubes for increasing number of components, from 1 to 10. We chose 10 as the max number of components because we saw a dramatic decline in contrast sensitivity after the first few components. In our analysis we employ the STIM largest intensity mask map \citep[SLIM map;][]{pairet_signal_2020} as an ensemble statistic. The SLIM map calculates the average STIM map from many residual cubes along with the average mask of the $N$ most intense pixels in each STIM map. A real companion ought to be present in many different residual cubes from the same dataset, so this ensemble approach can give us a probability map without predetermining the number of components. The collapsed residual frames, average STIM map, SLIM map, and contrast curves for each epoch for each of the above algorithms can be found in \autoref{sec:adi-results}. 

All of the ADI algorithms and metrics are implemented in the open-source Julia package ADI.jl \citep{lucas_adijl_2020}. All of the code for the ADI processing in this paper, including the scripts for each figure produced are in a GitHub repository\footnote{\href{https://github.com/mileslucas/sirius-b}{https://github.com/mileslucas/sirius-b}} in Jupyter notebooks and Julia scripts.

\section{Results} \label{sec:results}

\begin{figure*}
    \centering
    \epsscale{0.95}
    \plotone{residuals}
    \caption{The flat residuals of each epoch after PSF subtraction, derotating, and collapsing. The inner full-width at half-maximum (FWHM) is masked out for each frame.}
    \label{fig:residuals}
\end{figure*}

\begin{figure*}
    \centering
    \epsscale{0.95}
    \plotone{sig}
    \caption{The \textit{significance} maps for each epoch accounting for small sample statistics \citep{mawet_fundamental_2014}. Typically a critical value for detection is 5. The inner full-width at half-maximum (FWHM) is masked out for each map.}
    \label{fig:sig}
\end{figure*}

\begin{figure*}
    \centering
    \epsscale{0.95}
    \plotone{stim}
    \caption{The STIM maps for each epoch calculated from the residual cube. The STIM probability has a typical cutoff threshold of 0.5 for significant detections. The inner full-width at half-maximum (FWHM) is masked out for each map.}
    \label{fig:stim}
\end{figure*}

\begin{figure*}[t]
    \centering
    \epsscale{1}
    \plotone{contrast_curves}
    \caption{The contrast curves for the best performing algorithm from each epoch. The solid lines are the Gaussian 5$\sigma$ contrast curves and the dashed lines are the Student-t corrected curves. In addition, the expected upper limit for orbital separation of a stable orbit of \qty{1.5}{\au} is plotted as a vertical dashed line. The companion mass values are interpolated from the AMES-Cond grid. The lower mass limit (upper magnitude limit) of these models is plotted in a light-gray horizontal dashed line. The annular PCA curve is cut off because the contrast in the innermost annulus was greater than 1.}
    \label{fig:contrast}
\end{figure*}

We determined the best-performing algorithms for each epoch using the contrast curves described in \autoref{sec:analysis}. For the first two epochs full-frame median subtraction had the best contrast at almost all separations. For the last epoch annular PCA subtraction with 2 principal components and a rotation threshold of 0.5 FWHM produced the best contrast at close separations (\qtyrange{0.2}{0.4}{\arcsecond}) and had similar performance to other algorithms beyond \ang{;;0.4}. The innermost annulus from this algorithm has greater than 1 contrast and is not shown. The collapsed residual frames from each epoch are shown in \autoref{fig:residuals}, along with the Gaussian significance maps (\autoref{fig:sig}) and STIM maps (\autoref{fig:stim}).

The reduced images do not show consistent or overwhelming evidence for a substellar companion. The STIM probability maps for the 2020-11-21 and 2020-11-28 epochs suggest evidence for some blobs $\sim$\ang{;;0.13} from the center. The lack of evidence in the February epoch and the significance maps as well as the proximity to the central star ($\sim$2 FWHM) all reduce the probability of these blobs being true companions. Nonetheless, we estimated astrometry for blobs from each epoch (\autoref{tbl:astrometry}) and tried fitting Keplerian orbits using the ``Orbits for the Impatient'' algorithm \citep[OFTI;][]{blunt_orbits_2017}. We generated $10^4$ orbits, none of which managed to contain the points from each epoch (\autoref{sec:orbits}). We take this as direct evidence against the blobs being substellar companions of any kind.

It is interesting to note the morphology of the innermost $\sim$\ang{;;0.4} in the frames produced by GreeDS and NMF. Both of these algorithms outperform traditional median and PCA subtraction for disk imaging. In the frames from each epoch, but particularly in the two November epochs, a symmetric ``barbell'' shape can be seen which is similar to other disk images \citep[e.g., fig.~7][]{norris_vampires_2014}. Due to the nature of high-contrast imaging, it is difficult to differentiate systematic noise from real signal in the speckle-limited regime, in addition there is no prior evidence for a circumstellar disk from IR excess. Follow-up work in the visible (e.g., Subaru/VAMPIRES, VLT/SPHERE) may be able to image such a disk.

The contrast maps from the best performing algorithm for each reduction are shown in \autoref{fig:contrast}. We determine the limiting sensitivities in terms of the planetary mass by first calculating the contrast-limited magnitude using an Lp-band magnitude for Sirius B of 9.1 (adapted from \citealp{bonnet-bidaud_adonis_2008}). Then we use an age of \qty{226}{\mega\year} to interpolate the planetary mass using the AMES-Cond evolutionary grid and atmosphere models \citep{allard_models_2012}. It is important to note how the precision of the system age (see \autoref{sec:intro}) reduces uncertainty when interpolating planetary mass from the evolutionary grids. The best performing epoch was on the night of 2020-11-21, which managed to reach an exceptional sensitivity of \qty{3.5}{\jupitermass} at \qty{0.25}{\au} (\ang{;;0.09}) in the speckle limited regime and ultimately \qty{0.6}{\jupitermass} at \qty{1.5}{\au} (\ang{;;0.38}) in the sky-background limited regime.


\section{Conclusions} \label{sec:conclusion}

In closing, the Sirius system is one of the most well studied in history, with Sirius B being the target of companion searches from the visible to the IR. While it is highly unlikely a first-generation planet survived post-MS evolution, imaging efforts have gradually increased the sensitivity to second-generation planets. In this work we present high-contrast images of Sirius B in the near-IR. Our sensitivity limits are the best that have been reached for Sirius B, reaching \qty{0.6}{\jupitermass} at \qty{1.5}{\au}. In the mid-IR previous works have only probed as close-in as $\sim$\qty{1}{\au}, where we reach a sensitivity of \qty{0.72}{\jupitermass}. Despite the high sensitivity of this study, we found no appreciable evidence for a companion around Sirius B, consistent with previous results. This non-detection is compelling for revising occurrence rates of second-generation planets and debris disks around white dwarfs.

In recent years improving contrast and inner working angle were the dominant parameters of interest. Our observations here show that for nearby faint targets (white dwarfs, M-dwarfs, etc.) we are limited more by the sky background than the speckle noise. Our sensitivities reach the background limit at $\sim$\ang{;;0.4}, which would correspond to projected separations of \qtyrange{1}{10}{\au} for stars at \qtyrange{2.7}{25}{\parsec}. This means potentially reach sub-Jupiter sensitivities around young systems ($\lesssim$\qty{300}{\mega\year}) and $<$\qty{10}{\jupitermass} sensitivities around moderate age systems ($\lesssim$\qty{7}{\giga\year}) for 10s-100s of targets without coronagraphy or extreme AO.

We have published alongside this work the entire codebase used for pre-processing and reducing the data, and for generating every figure in this manuscript. We have also published our reduced datasets under an open license. We hope that this improves the reproducibility of the work as well as providing data for exploring new and different ADI algorithms.

\begin{acknowledgments}

\end{acknowledgments}

\software{
ADI.jl \citep{lucas_adijl_2020},
astropy \citep{collaboration_astropy_2013,astropy_collaboration_astropy_2018},
Julia \citep{bezanson_julia_2017},
numpy \citep{harris_array_2020},
scikit-image \citep{walt_scikit-image_2014},
}

\bibliography{references}{}
\bibliographystyle{aasjournal}

\appendix

\section{ADI Processing Results} \label{sec:adi-results}


\begin{figure}[h!]
    \centering
    \plotone{2020feb04_contrast_curves}
    \caption{5$\sigma$ contrast curves from every ADI algorithm for the first epoch. Both the Gaussian (solid lines) and Student-t corrected (dashed lines) contrast curves are shown. }
\end{figure}

\begin{figure}[h!]
    \centering
    \plotone{2020nov21_contrast_curves}
    \caption{5$\sigma$ contrast curves from every ADI algorithm for the second epoch. Both the Gaussian (solid lines) and Student-t corrected (dashed lines) contrast curves are shown. }
\end{figure}

\begin{figure}[h!]
    \centering
    \plotone{2020nov28_contrast_curves}
    \caption{5$\sigma$ contrast curves from every ADI algorithm for the third epoch. Both the Gaussian (solid lines) and Student-t corrected (dashed lines) contrast curves are shown. }
\end{figure}

%%%%%%%%%%%%%%%%%%%%%%%%%%%%%%%%%%%%%%%%%%%%%%%%

%%%%%%%%%%%%%%%%%%%%%%%%%%%%%%%%%%%%%%%%%%%%%%%%

\begin{figure}[h!]
    \centering
    \figurenum{10}
    \plotone{reports/2020feb04_median}
    \caption{Post-processing results from the second epoch using full-frame median subtraction. The top-left frame is the collapsed residual frame, the top-right is the Gaussian S/N map, the bottom-left is the STIM probability map, and the bottom-right is the Student-t corrected significance map. In each frame, the inner two FWHMs are masked out. The right figure show the Gaussian (solid line) and Student-t corrected (dashed curve) 5$\sigma$ contrast curve. Outputs for other epochs and other algorithms (21 figures) are in the online figure set and the GitHub repository.}
\end{figure}

%%%%%%%%%%%%%%%%%%%%%%%%%%%%%%%%%%%%%%%%%%%%%%%%

%%%%%%%%%%%%%%%%%%%%%%%%%%%%%%%%%%%%%%%%%%%%%%%%

\begin{figure}[h!]
    \centering
    \figurenum{11}
    \plotone{reports/2020feb04_pca_mosaic}
    \caption{Collapsed residual frames from the first epoch using PCA reduction with 1-10 components. The figures share a common scale and the inner two FWHMs are masked out for all the frames. Outputs for the other epochs and the NMF and GreeDS algorithms (9 figures) are in the online figure set and the GitHub repository}
\end{figure}

%%%%%%%%%%%%%%%%%%%%%%%%%%%%%%%%%%%%%%%%%%%%%%%%

%%%%%%%%%%%%%%%%%%%%%%%%%%%%%%%%%%%%%%%%%%%%%%%%

\begin{figure}[h!]
    \centering
    \figurenum{12}
    \plotone{reports/2020feb04_pca_contrast_curves}
    \caption{5$\sigma$ Gaussian contrast curves for the first epoch using PCA reduction with 1-10 components. The left two figures are the STIM probability map and the SLIM detection map. For both of these maps, a typical cutoff value is 0.5. Outputs for the other epochs and the NMF and GreeDS algorithms (9 figures) are in the online figure set and the GitHub repository.}
    \label{fig:pca-contrast-curves}
\end{figure}


\clearpage
\section{Provisional Orbit Fitting} \label{sec:orbits}

We found multiple interesting blobs in the reduced data that were not statistically significant. Nonetheless, we tried fitting Keplerian orbits using OFTI to determine the feasibility of the blobs being real companions. We began by estimating the astrometry of the blobs by eye in reduced data (\autoref{tbl:astrometry}, \autoref{fig:prov-orbit}). We tried simulating $10^4$ orbits via rejection sampling with OFTI but failed to constrain all three points with any one orbit. Overall we determine these blobs are not real companions and are most likely systematic noise in the stellar PSF.

\begin{deluxetable}{ccc}[h!]
    \tablecaption{Provisional astrometry for blobs of interest from each epoch. The separation and offset are in relation to Sirius B. The uncertainties are derived from the FWHM of the PSF from each epoch.
    \label{tbl:astrometry}}
    \tablehead{
        \colhead{Date observed} &
        \colhead{offset (\unit{\milliarcsecond})} &
        \colhead{PA (\unit{\degree})}
    }
    \startdata
    2020-02-04 & $114\pm 40$ & $-115\pm 20$  \\
    2020-11-21 & $121\pm 38$ & $-119\pm 18$  \\
    2020-11-28 & $114\pm 41$ & $-113\pm 21$  \\
    \enddata
\end{deluxetable}

\begin{figure}[h!]
    \centering
    \figurenum{13}
    \plotone{orbit_frames}
    \caption{Provisional astrometry (white circles) displayed on STIM maps using the GreeDS algorithm with 2 components. Each frame was cropped to the inner \ang{;;0.25} and the inner FWHM has been masked out. The width of the circles represent the uncertainty.}
    \label{fig:prov-orbit}
\end{figure}

\end{document}
