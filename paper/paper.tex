\documentclass[twocolumn]{aastex631}
\usepackage{hyperref}
\let\tablenum\relax
\usepackage[output-exponent-marker = \text{e}]{siunitx}

%% Define new commands here
\newcommand\latex{La\TeX}

\graphicspath{{./}{figures/}}
\shortauthors{Lucas et al.}

\begin{document}

\title{A Search for Circumstellar Companions Around Nearby White Dwarf Sirius b}

\correspondingauthor{Miles Lucas}
\email{mdlucas@hawaii.edu}

\author[0000-0001-6341-310X]{Miles Lucas}
\affiliation{Institute for Astronomy, University of Hawai'i, USA}

\author[0000-0003-1341-5531]{Michael Bottom}
\affiliation{Institute for Astronomy, University of Hawai'i, USA}

\author[0000-0003-4769-1665]{Garreth Ruane}
\affiliation{Jet Propulsion Laboratory, California Institute of Technology, USA}

\author[0000-0002-0696-1780]{Sam Ragland}
\affiliation{W.M. Keck Observatory, USA}


\begin{abstract}

\end{abstract}

\section{Introduction}

Directly imaging exoplanets is a powerful method in exoplanet science which explores a parameter space very complementary to the popular transit photometry and radial velocity methods. Two of the key parameters that limit direct imaging are \textit{contrast}, the relative brightness between the host star and the companion, and \textit{separation}, due to the diffraction and seeing limited nature of astronomy, especially the effects of Earth's atmosphere above ground-based telescopes. Sirius B is an enticing prospect for a direct imaging search because it is very close, only \SI{2.7}{pc} away, which allows probing very close-in orbital separations, down to a few AU. Also, due to the strong cooling of white dwarfs, the luminosity will be 3-4 orders of magnitude less than its main sequence progenitor. This means current ground-based instrumetns can probe 3-4 orders of magnitude deeper into the planetary mass space.

Any potential companion around an evolved star must have survived multiple stages of stellar evolution, notably the red giant phase and the asymptotic giant phase. In the red giant phase, the main sequence star inflates in a shell around the collapsing inert core. Any planet within the Roche limit of the expanding star will be engulfed, polluting the stellar atmosphere. In the asymptotic giant phase, the star will be blowing off mass in shells which can become highly ionized by strong ultraviolet and X-ray emission from the star, which has consequences for planetary atmospheres. By the end of the star's evolution, as it enters the white dwarf cooling sequence, it is expected that a planet beyond a critical separation (to escape engulfment during the red giant phase) will typically migrate outward as a function of the mass loss of the star.

\section{Observations}

Sirius is the closest star system to the Sun at \SI{2.7}{pc} away. The host, Sirius a is infamously one of the brightest stars in the sky, with a K-band magnitude of -1.35. Sirius b has a well-studied orbit

\section{Analysis}

For each epoch


% \acknowledgments

% \software{
% astropy \citep{2013A&A...558A..33A,2018AJ....156..123A},
% numpy \citep{harris2020array},
% scikit-image \citep{2014arXiv1407.6245V},
% }

\bibliography{references}{}
\bibliographystyle{aasjournal}

\end{document}
