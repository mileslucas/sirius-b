\documentclass[twocolumn]{aastex631}
\usepackage{hyperref}
\let\tablenum\relax
\usepackage[output-exponent-marker = \text{e}]{siunitx}


\DeclareSIUnit\parsec{pc}
\DeclareSIUnit\au{AU}
\DeclareSIUnit\mj{M_J}
\DeclareSIUnit\year{yr}

%% Define new commands here
\newcommand\latex{La\TeX}

\graphicspath{{./}{figures/}}
\shortauthors{Lucas et al.}

\begin{document}

\title{A Search for Circumstellar Companions Around Nearby White Dwarf Sirius b}

\shorttitle{A Search for Circumstellar Companions around Sirius b}
\shortauthors{Lucas et al.}

\correspondingauthor{Miles Lucas}
\email{mdlucas@hawaii.edu}

\author[0000-0001-6341-310X]{Miles Lucas}
\affiliation{Institute for Astronomy, University of Hawai'i, USA}

\author[0000-0003-1341-5531]{Michael Bottom}
\affiliation{Institute for Astronomy, University of Hawai'i, USA}

\author[0000-0003-4769-1665]{Garreth Ruane}
\affiliation{Jet Propulsion Laboratory, California Institute of Technology, USA}

\author[0000-0002-0696-1780]{Sam Ragland}
\affiliation{W.M. Keck Observatory, USA}


\begin{abstract}

\end{abstract}

\section{Introduction} \label{sec:intro}

Very little is known about the evolution of planetary systems beyond the main-sequence lifetime of their host stars. Theoretical calculations show that planets can survive the red giant and asymptotic giant stages \citep{burleigh_imaging_2002,jura_pollution_2008,nordhaus_orbits_2013}, and results from radial velocity and transit surveys suggest extrasolar planets are prevalent throughout the Milky Way \citep{cumming_keck_2008}. However, only one planetary mass object has been detected around a white dwarf (WD) with imaging \citep[WD 0806-661;][]{luhman_discovery_2011}.

Directly imaging exoplanets is a powerful method in exoplanet science which explores a parameter space very complementary to the popular transit photometry and radial velocity methods. The red giant stage has cleared planets within a few \si{au} greatly lowering the transit probability. White dwarfs have very few spectral lines to make high-precision radial velocity measurements from spectral lines. Imaging, though, benefits from the orbital expansion of the planet, and the cooling of the white dwarf greatly reduces the contrast between them \citep{burleigh_imaging_2002,gould_finding_2008}. The Degenerate Objects around Degenerate Objects (DODO) survey observed 29 white dwarfs with Gemini/NIRI and VLT/ISAAC and provided upper limits around $\sim$\SI{8}{\mj} beyond \SI{35}{\au} \citep{hogan_dodo_2009}.

In this work we study the closest white dwarf to the Sun, Sirius b, and provide upper mass limits based on our observations in the near-infrared. \autoref{sec:obs} describes our target and observing strategy, \autoref{sec:analysis} describes the processing and analysis techniques used, \autoref{sec:results} describes our results and we conclude with \autoref{sec:conclusion}.

\section{Observations} \label{sec:obs}

Sirius is the closest star system to the Sun at \SI{2.7}{\parsec} away, consisting of an A1Vm star (a) and a WD with an age around \SI{225}{\mega\year} \citep{bond_sirius_2017,collaboration_gaia_2018}. There are no known planets around either star, and using constrained 3-body numerical simulations a potential companion around Sirius b would only be stable with a \SI{1.79}{\year} orbit, which corresponds to a circular orbit with \SI{1.5}{\au} separation. Sirius b has been studied in the \SI{10}{\micro\meter} regime recently by \cite{pathak_high_2021}, who achieve \SIrange[]{1.8}{3.5}{\mj} upper limits within 2 arcseconds.

We targeted Sirius b directly using Keck/NIRC2 in Lp-band across three epochs in 2020 \autoref{tbl:obs}. Due to the high proper motion and reasonable orbital motion, we started by calculating the projected offset and position angle from Sirius a based on the orbital parameters from \cite{bond_sirius_2017}. We first slewed to Sirius a, which is easy to lock onto due to its brightness \citep[-1.35 K-band magnitude;][]{bonnet-bidaud_adonis_2008}. We found that Sirius b is too dim \citep[9.01 K-band magnitude;][]{bonnet-bidaud_adonis_2008} for the adaptive optics (AO) performance necessary for coronagraphy and that the scattered light from Sirius a was detrimental to closing the AO loop when using Sirius b as a natural guide star (NGS) anyways. Therefore, we employed a unique off-axis solution, using a narrow laser-line filter for the wavefront sensor using Sirius a as the NGS and offsetting to Sirius b manually.

The scattered light from Sirius a is so strong, in fact, that radial diffraction spikes from the Keck II secondary mirror support structure are still present in the Sirius b frames, which reduced our overall observing efficiency due to discarding these frames. An example of such a diffraction spike is shown in \autoref{fig:spike}.

\begin{figure}
    \centering
    \epsscale{1}
    \plotone{spike}
    \caption{Calibrated science frame of Sirius b from 2020-02-04 epoch showing the strong scattered light effects from Sirius a.}
    \label{fig:spike}
\end{figure}


\begin{deluxetable*}{ccccccc}
    % \tabletypesize{\scriptsize}
    \tablecaption{Observing parameters for the three epochs of data. All observations were carried out using the NIRC2 Lp-band filter. Observation time is based on the frames that were selected for processing.
    \label{tbl:obs}}
    \tablehead{
        \colhead{Date observed} &
        \colhead{Sirius a offset (")} &
        \colhead{Sirius a PA (\textdegree)} &
        \colhead{Observation time (hr)} & 
        \colhead{Seeing (arcsec)} & 
        \colhead{Temp (\textdegree C)} & 
        \colhead{PWV (mm)}
    }
    \startdata
    2020-02-04 & 11.20 & 67.90 & 1.44 &  &  &  \\
    2020-11-21 & 11.27 & 66.42 & 2.91 &  &  &  \\
    2020-11-28 & 11.27 & 66.38 & 2.44 &  &  &  \\
    \enddata
\end{deluxetable*}

\section{Analysis} \label{sec:analysis}

For each epoch we applied a flat correction using calibration frames captured during observing. We also removed bad pixels using a combination of L.A.Cosmic \citep{dokkum_cosmic-ray_2001} and an adaptive sigma-clipping algorithm. We finally remove sky background using a high-pass median filter. At this point frames were manually filtered to remove bad frames, especially those with diffraction spikes like in \autoref{fig:spike}. Then, each good frame is co-registered using cross-correlations \citep{guizar-sicairos_efficient_2008} and aligned with the center of the FOV. Lastly the frames were cropped to the inner 200 pixels and stacked into data cubes for each epoch. With the pixel scale of \SI{10}{mas\per px} the crop corresponds to a maximum separation of \SI{2.7}{\au}. We also measure the parallactic angle of each frame, including corrections for distortion effects following \cite{yelda_improving_2010}. For each epoch, we measure the full-width at half-maximum (FWHM) of the stellar PSF by fitting a bivariate Gaussian model to the median frame from each data cube (an example of one is shown in \autoref{fig:psf}).

\begin{figure}
    \centering
    \epsscale{1}
    \plotone{psf}
    \caption{The median frame from the 2020-11-21 epoch showing the instrumental PSF. The inner core has a FWHM of $\sim$\SI{76}{mas}. The speckle pattern can be seen in the blobs surrounding the first ring, with roughly 6-way radial symmetry corresponding to the hexagonal shapes of the segmented mirrors.}
    \label{fig:psf}
\end{figure}

During observing we applied the technique of angular differential imaging \citep[ADI;][]{marois_angular_2006} where we disable the field rotator and allow the FOV to rotate throughout the night along with Earth. The point-spread function (PSF), which is an instrumental effect, will not appear to rotate while any potential companion will appear to rotate. This allows modeling and subtracting the PSF with less probability for subtracting companion signal. After subtraction, the frames are derotated by their parallactic angle and combined with a weighted sum \citep{bottom_noise-weighted_2017}, which reduces the pixel-to-pixel noise as the number of frames in the data cube increases.

For this analysis we used four statistical models for modeling the stellar PSF, classic median subtraction \citep{marois_angular_2006}, principal component analysis \citep[PCA;][]{soummer_detection_2012}, non-negative matrix factorization\citep[NMF;][]{ren_non-negative_2018}, and fixed-point greedy disk subtraction \citep[GreeDS;][]{pairet_reference-less_2019,pairet_mayonnaise_2020}. The median subtraction and PCA methods were also applied in an annular method, where we model the PSF in annuli of increasing separation, since each annulus has similar statistcal properties under the framework of ADI \citep{marois_angular_2006}.

We used three metrics for determining the performance of each algorithm, the signal-to-noise ratio (S/N) detection map, the standardized trajectory intensity mean map \citep[STIM map;][]{pairet_stim_2019}, and the STIM largest intensity mask map \citep[SLIM map;][]{pairet_signal_2020}. These metrics assign a likelihood to each pixel for the presence of a companion using different assumptions of the residual statistics. The SLIM map is an ensemble statistic which calculates the average STIM map from many residual cubes, such as many variations of PCA with differing number of components.

Another metric we calculate for some of the algorithms is the contrast curve, which determines the sensitivity of a 5$\sigma$ statistical detection through repeated injection and retrieval of planetary signal as processed by one of the ADI algorithms above. We calculate both the Gaussian contrast and the Student-t corrected contrast, which accounts for the small-sample statistics in each annulus \citep{mawet_fundamental_2014}.

\section{Results} \label{sec:results}

\begin{figure*}
    \centering
    \epsscale{1.1}
    \plotone{residuals}
    \caption{The flat residuals of each epoch after PSF subtraction, derotating, and collapsing. The inner full-width at half-maximum (FWHM) is masked out for each frame.}
    \label{fig:residuals}
\end{figure*}

\begin{figure*}
    \centering
    \epsscale{1.1}
    \plotone{snr}
    \caption{The S/N maps for each epoch calculated with Gaussian noise assumptions. The inner full-width at half-maximum (FWHM) is masked out for each map.}
    \label{fig:snr}
\end{figure*}

\begin{figure*}
    \centering
    \epsscale{1.1}
    \plotone{stim}
    \caption{The STIM maps for each epoch calculated from the residual cube. Note that the STIM probability has a typical cutoff threshold of 0.5 for significant detections. The inner full-width at half-maximum (FWHM) is masked out for each map.}
    \label{fig:stim}
\end{figure*}


The best performing algorithms for each epoch in terms of sensitivity were median subtraction median subtraction, and PCA subtraction with 2 principal components. The flat residual images from each epoch are shown in \autoref{fig:residuals}, along with the Gaussian S/N maps (\autoref{fig:snr}) and STIM maps (\autoref{fig:stim}). For each epoch we determined there was not significant evidence for a companion around Sirius b.



\section{Conclusions} \label{sec:conclusion}


\begin{acknowledgments}

\end{acknowledgments}

\software{
ADI.jl \citep{lucas_adijl_2020},
astropy \citep{collaboration_astropy_2013,astropy_collaboration_astropy_2018},
Julia \citep{bezanson_julia_2017},
numpy \citep{harris_array_2020},
scikit-image \citep{walt_scikit-image_2014},
}

\bibliography{references}{}
\bibliographystyle{aasjournal}

\appendix

\section{ADI Processing Results}


\begin{figure}[h!]
    \centering
    \plotone{2020feb04_contrast_curves}
    \caption{5$\sigma$ contrast curves from every ADI algorithm for the first epoch. Both the Gaussian (solid lines) and Student-t corrected (dashed lines) contrast curves are shown. }
\end{figure}

\begin{figure}[h!]
    \centering
    \plotone{2020nov21_contrast_curves}
    \caption{5$\sigma$ contrast curves from every ADI algorithm for the second epoch. Both the Gaussian (solid lines) and Student-t corrected (dashed lines) contrast curves are shown. }
\end{figure}

\begin{figure}[h!]
    \centering
    \plotone{2020nov28_contrast_curves}
    \caption{5$\sigma$ contrast curves from every ADI algorithm for the third epoch. Both the Gaussian (solid lines) and Student-t corrected (dashed lines) contrast curves are shown. }
\end{figure}

%%%%%%%%%%%%%%%%%%%%%%%%%%%%%%%%%%%%%%%%%%%%%%%%

%%%%%%%%%%%%%%%%%%%%%%%%%%%%%%%%%%%%%%%%%%%%%%%%

\begin{figure}[h!]
    \centering
    \figurenum{10}
    \plotone{reports/2020feb04_median}
    \caption{Post-processing results from the second epoch using full-frame median subtraction. The top-left frame is the collapsed residual frame, the top-right is the Gaussian S/N map, the bottom-left is the STIM probability map, and the bottom-right is the Student-t corrected significance map. In each frame, the inner two FWHMs are masked out. The right figure show the Gaussian (solid line) and Student-t corrected (dashed curve) 5$\sigma$ contrast curve. Outputs for other epochs and other algorithms (21 figures) are in the online figure set and the GitHub repository.}
\end{figure}

%%%%%%%%%%%%%%%%%%%%%%%%%%%%%%%%%%%%%%%%%%%%%%%%

%%%%%%%%%%%%%%%%%%%%%%%%%%%%%%%%%%%%%%%%%%%%%%%%

\begin{figure}[h!]
    \centering
    \figurenum{11}
    \plotone{reports/2020feb04_pca_mosaic}
    \caption{Collapsed residual frames from the first epoch using PCA reduction with 1-10 components. The figures share a common scale and the inner two FWHMs are masked out for all the frames. Outputs for the other epochs and the NMF and GreeDS algorithms (9 figures) are in the online figure set and the GitHub repository}
\end{figure}

%%%%%%%%%%%%%%%%%%%%%%%%%%%%%%%%%%%%%%%%%%%%%%%%

%%%%%%%%%%%%%%%%%%%%%%%%%%%%%%%%%%%%%%%%%%%%%%%%

\begin{figure}[h!]
    \centering
    \figurenum{12}
    \plotone{reports/2020feb04_pca_contrast_curves}
    \caption{5$\sigma$ Gaussian contrast curves for the first epoch using PCA reduction with 1-10 components. The left two figures are the STIM probability map and the SLIM detection map. For both of these maps, a typical cutoff value is 0.5. Outputs for the other epochs and the NMF and GreeDS algorithms (9 figures) are in the online figure set and the GitHub repository.}
    \label{fig:pca-contrast-curves}
\end{figure}



\begin{figure*}
    \plotone{2020feb04_contrast_curves}
    \caption{5$\sigma$ contrast curves from various ADI algorithms for the 2020-02-04 epoch. Both the Gaussian (solid lines) and Student-t corrected (dashed lines) contrast curves are shown.}
\end{figure*}

\begin{figure*}
    \plotone{2020nov21_contrast_curves}
    \caption{5$\sigma$ contrast curves from various ADI algorithms for the 2020-11-21 epoch. Both the Gaussian (solid lines) and Student-t corrected (dashed lines) contrast curves are shown.}
\end{figure*}

\begin{figure*}
    \plotone{2020nov28_contrast_curves}
    \caption{5$\sigma$ contrast curves from various ADI algorithms for the 2020-11-28 epoch. Both the Gaussian (solid lines) and Student-t corrected (dashed lines) contrast curves are shown.}
\end{figure*}

\end{document}
