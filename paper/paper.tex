\documentclass[twocolumn]{aastex631}
\usepackage{hyperref}
\let\tablenum\relax
\usepackage[output-exponent-marker = \text{e}]{siunitx}


\DeclareSIUnit\parsec{pc}
\DeclareSIUnit\au{AU}
\DeclareSIUnit\mj{M_J}
\DeclareSIUnit\ms{M_\odot}
\DeclareSIUnit\year{yr}
\sisetup{
    list-units=single
}

%% Define new commands here
\newcommand\latex{La\TeX}

\graphicspath{{./}{figures/}}
\shortauthors{Lucas et al.}

\begin{document}

\title{A Search for Circumstellar Companions Around Nearby White Dwarf Sirius b}

\shorttitle{A Search for Circumstellar Companions around Sirius b}
\shortauthors{Lucas et al.}

\correspondingauthor{Miles Lucas}
\email{mdlucas@hawaii.edu}

\author[0000-0001-6341-310X]{Miles Lucas}
\affiliation{Institute for Astronomy, University of Hawai'i, USA}

\author[0000-0003-1341-5531]{Michael Bottom}
\affiliation{Institute for Astronomy, University of Hawai'i, USA}

\author[0000-0003-4769-1665]{Garreth Ruane}
\affiliation{Jet Propulsion Laboratory, California Institute of Technology, USA}

\author[0000-0002-0696-1780]{Sam Ragland}
\affiliation{W.M. Keck Observatory, USA}


\begin{abstract}

We present the deepest study of planetary companions around Sirius B, the closest and brightest white dwarf in the night sky. We use Keck/NIRC2 in Lp band (\qty{3.776}{\micro\meter}) across 3 epochs in 2020 to take direct images of Sirius B. Each epoch was prepared and processed using the technique of angular differential imaging, which allowed us to reach exceptional sensitivity limits. The best-performing contrast from each epochs reached sensitivities of \qtylist{3.5;1.9;1.1;0.72;0.63}{\mj} at projected separations of \qtylist{0.25;0.5;0.75;1;2}{\au} (\qtylist{0.09;0.19;0.28;0.38;0.75}{\arcsecond}) respectively. These sensitivites are the best yet around Sirius B and are better than any limits reached around other evolved star. Despite the exceptional sensitivity of our observations, we do not report any additional companions around Sirius B.

\end{abstract}

\section{Introduction} \label{sec:intro}

High-contrast imaging (HCI) is a powerful technique for discovering and characterizing exoplanets. Being able to probe the architecture, formation, and atmospheres of planets directly is necessary for advancing substellar companion formation and evolution theory. The process required to image a planet is daunting, however. The typical astrophysical flux ratios for a Sun-Jupiter analog in the near-infrared (NIR) are $\sim10^{-8}$, and for a Sun-Earth system are $\sim10^{-10}$ \citep{traub_direct_2010}. In addition, the close angular separation of planets makes it difficult to distinguish them from the diffraction pattern of the star and other noise sources. In order to make such a difficult detection it is imperative to minimize diffraction, scattering, and systematic noise.

Nearby white dwarfs (within $\sim$\qty{100}{\parsec}) are compelling targets for direct imaging searches, despite the limited knowledge of planetary systems around evolved stars. \cite{burleigh_imaging_2002} suggests planets on initially wide ($>$\qty{5}{\au}) orbits around intermediate mass stars (\qtyrange{1}{8}{\ms}) will survive expansion during the red giant (RGB) phase and remain bound to the resultant white dwarf. Furthermore, \citet{nordhaus_orbits_2013} explain that close-in planets that escape the Roche limit of the expanding red giant can still be shredded by tidal forces, however the tidal forces on planets that readily escape engulfment are negligible. During the asymptotic giant (AGB) phase, strong ionizing soft X-ray and UV emission will likely photoevaporate planetary atmospheres. In addition, the mass loss of the star will adiabatically expand the orbit of the planet by a maximum factor of $M_{*,\mathrm{MS}}/M_{*,\mathrm{WD}}$ \citep{jeans_cosmogonic_1924}.

If the planet has managed to survive through the RGB and AGB phases, stars below \qty{8}{\ms} will enter the white dwarf cooling sequence, which will lead to a reduction in luminosity by $10^3-10^4$. This means the contrast between a white dwarf and a companion will be 3-4 orders of magnitude lower than the equivalent main sequence system. The combination of expanded orbits and lower contrast are beneficial for imaging, due to the limited inner working angles and sensitivities of ground-based instrumentation. The transit photometry method is very insensitive to planets around white dwarfs because the probability of a transit decreases with increasing separation. Similarly, the radial velocity method is limited to very massive planets on larger separations in addition to the extremely broad spectral features of a white dwarf which greatly limit the precision of radial velocity measurements.

The first search for substellar companions around white dwarfs was conducted by \citet{probst_infrared_1983} by looking for infrared (IR) excess in their spectral energy distributions (SED) using broadband photometry. This excess could likely be attributed to circumstellar dust or gas discs, which are indicators of the planetary formation history. Of the $\sim$ 100 white dwarfs they studied, no companions were found. The same method was applied by \citet{zuckerman_excess_1987} who found excess IR emission around white dwarf G29-38. Eventually this excess was determined to be from a dust disc \citep{telesco_observations_1990}, with the current interpretation associating the dust with accretion onto the white dwarf, polluting the stellar atmosphere \citep{koester_metals_1997}.

The only planetary-mass object imaged around a single white dwarf (WD 0806-661B) was discovered using Spitzer with a mass of \qty{7}{\mj} on a very wide \qty{2500}{\au} orbit \citep{luhman_discovery_2011}. The ``Degenerate Objects around Degenerate Objects'' survey \citep[DODO;][]{hogan_dodo_2009} observed 29 white dwarfs with Gemini/NIRI and VLT/ISAAC, reaching an average upper limit around $\sim$\qty{8}{\mj} beyond \qty{35}{\au}. The young white dwarf GD 50 was observed using the extreme AO instrument SPHERE at the VLT \citep{xu_extreme-ao_2015}, reaching sensitivity limits of \qty{4}{\mj} at \qty{6.2}{\au}. Sirius B has also been imaged, which will be discussed later. To date, there has been no direct images of an exoplanet around a white dwarf.

The Sirius system is the 7th closest to the sun at \qty{2.7}{\parsec}, consisting of Sirius A, an A1Vm star known for being the brightest in the night sky, and Sirius B, a DA2 white dwarf with a \qty{50}{\year} orbit \citep{bond_sirius_2017,collaboration_gaia_2018}. The proximity and brightness of the Sirius system make it one of the most well-studied systems throughout history \citep[see][]{wesemael_sirius_1982}.

The orbit of Sirius B was first studied by \citet{bessel_variations_1844}, who recognized wobbles in the proper motion of Sirius A caused by a ``dark satellite''. \citet{bond_companion_1862} reported and confirmed visual evidence of Sirius B by Alvan G. Clark and his father. \citet{adams_spectrum_1915} took the first spectral measurements of Sirius B and found it to be very similar to a main-sequence early A-type star despite the faintness, which we know know to be typical of white dwarf specra. 

Initial astrometric perturbations suggested a 50 year orbital period \citep{auwers_orbit_1864}, with the first orbital solution including dynamic masses published by \citet{van_den_bos_orbit_1960}. \citet{gatewood_study_1978} improve upone the van den Bos orbital solution with over 300 more images, finding nearly the same dynamical masses for the binary. \citet{bond_sirius_2017} greatly refine the orbit using a compilation of historical data and HST data, which gives dynamical masses of \qty{2.063+-0.023}{\ms} and \qty{1.018+-0.011}{\ms} for A and B, respectively. 

The precise dynamical masses are key for finding the age of Sirius; \citet{bond_sirius_2017} report a cooling age for Sirius B of \qty{126}{\mega\year} using isochrones. Studies of the initial-final mass relation (IFMR) of white dwarfs \citep{cummings_two_2016} lead to estimates of the progenitor mass of Sirius B to be between \qtyrange{5}{5.6}{\ms}, which when combined with stellar evolution codes yield total ages between \qtyrange{226}{228}{\mega\year} with an uncertainty of about $\pm$\qty{10}{\mega\year} \citep{bond_sirius_2017}.

A companion around Sirius B would be affected by the orbit of Sirius A, and this constrained three-body system has been studied numerically \citep{holman_long-term_1999}. \citet{bond_sirius_2017} calculate the longest period stable companion around Sirius B is \qty{1.79}{\year}, which corresponds to a \qty{1.5}{\au} circular orbit.

Some of the first modern studies searching for companions around Sirius B were \citet{schroeder_search_2000} and \citet{kuchner_search_2000}, who used the HST planetary camera and NICMOS cameras, respectively, looking for signal around \qty{1}{\micro\meter}, which combined had a sensitivity down to $\sim$\qty{10}{\mj} at \ang{;;2} (\qty{5.3}{\au}). \citet{bonnet-bidaud_adonis_2008} use the ground-based ESO/ADONIS instrument in JHKs to reach a sensitivty of $\sim$\qty{30}{\mj} at \ang{;;3} (\qty{7.9}{\au}). \citet{skemer_sirius_2011} used mid-IR (out to \qty{10}{\micro\meter}) observations from Gemini/T-ReCs which ruled out evidence for any infrared excess around Sirius B. \citet{thalmann_piercing_2011} use Subaru/IRCS at \qty{4.05}{\micro\meter} reaching detection sensitivities of \qtyrange{6}{12}{\mj} at \ang{;;1}. Very recently, \citet{pathak_high_2021} took simultaneous mid-IR observations (\qty{10}{\micro\meter}) at VLT/VISIR of Sirius A (through a coronagraph) and B. Because of the simultaneous observation, their contrast is split depending on which half of the FoV is being considered. Their average sensitivity is $\sim$\qty{2.5}{\mj} at $\sim$\qty{1}{\au}, and their best sensitivity (from the ``inner'' region) is $\sim$\qty{1.5}{\mj} at \qty{1}{\au}.

In this work we report our study of Sirius B using direct images from Keck/NIRC2. \autoref{sec:obs} describes our target and observing strategy, \autoref{sec:analysis} describes the processing and analysis techniques used, \autoref{sec:results} describes our results and we conclude with \autoref{sec:conclusion}.

\section{Observations} \label{sec:obs}

Despite Sirius B being the brightest white dwarf in the sky, it is still 10 magnitudes dimmer than Sirius A, which makes it it a very technically challenging target, especially on ground-based telescopes which have to deal with extra scaterring and diffraction from the atmosphere.

We targeted Sirius B directly using Keck/NIRC2 in Lp-band (\qty{3.776}{\micro\meter}) across three epochs in 2020 (\autoref{tbl:obs}). Our first attempt to observe Sirius B failed due to the strong scattered light from Sirius A. The adaptive optics (AO) calibration routinely failed when the extremely bright (-1.35 K-band magnitude) diffraction patterns would sweep into the field of view (FoV) of Sirius B (shown in \autoref{fig:spike}). Similarly, trying to deploy the focal-plane vortex coronagraph \citep{serabyn_w_2017} failed when the low-order wavefront correction algorithm, QACITS \citep{huby_w_2017}, performed erratically in the presence of the scattered light. This is very similar to the issues reported by \citet[\S2]{vigan_high-contrast_2015}  in their attempts to image Sirius B. In order to overcome these issues, we decided to try using Sirius A as the NGS and off-axis guiding to Sirius B.

The Keck AO system \citep{wizinowich_performance_2000} was not designed to accomadate stars as bright as Sirius A, so we needed to attenuate the flux greatly to avoid saturating the wavefront sensor (WFS). We accomplished this using a narrow laser-line filter in front of the WFS, which brought Sirius A from $\sim$ -1.35 mag down to $\sim$ 5 mag. While still extremely bright, this was enough attenuation to close the AO loop. From here, we slewed off-axis using the separations and position angles calculated in \autoref{tbl:obs} from the orbital parameters in \cite{bond_sirius_2017}. In this mode, we noticed higher than usual drift in the focal plane, requiring manually recentering the target every 5 or 10 minutes. For both the November epochs we tried exploiting this by dithering between two positions in order to simplify background subtraction, but this ended up performing worse than the high-pass filter method we used in \autoref{sec:analysis}. We tried deploying the vortex coronagraph along with QACITS, but again the scattered light from Sirius A made QACITS unstable, so we decided to forego any coronagraphy for the remaining observations.

During each observation, calibration frames were taken in the form of dark frames and sky flat frames. We also disabled the field rotator, which caused the FoV to rotate throughout the night for angular differential imaging \citep[ADI;][]{marois_angular_2006}.

\begin{figure}
    \centering
    \epsscale{1}
    \plotone{spike}
    \caption{Calibrated science frame of Sirius b from 2020-02-04 epoch showing the strong scattered light effects from Sirius a.}
    \label{fig:spike}
\end{figure}


\begin{deluxetable*}{ccccccccc}
    \tabletypesize{\scriptsize}
    \tablecaption{Observing parameters for the three epochs of data. All observations were carried out using the NIRC2 Lp-band filter. Observation time is based on the frames that were selected for processing.
    \label{tbl:obs}}
    \tablehead{
        \colhead{Date observed} &
        \colhead{Sirius A offset (arcsec)} &
        \colhead{Sirius A PA (\textdegree)} &
        \colhead{Observation time (hr)} & 
        \colhead{FoV rotation (\textdegree)} &
        \colhead{FWHM (mas)} &
        \colhead{Seeing (arcsec)} & 
        \colhead{Temp (\textdegree C)} & 
        \colhead{PWV (mm)}
    }
    \startdata
    2020-02-04 & 11.20 & 67.90 & 1.44 & 60.1 & 79.9 & &  &  \\
    2020-11-21 & 11.27 & 66.42 & 2.91 & 91.4 & 76.4 & &  &  \\
    2020-11-28 & 11.27 & 66.38 & 2.44 & 80.4 & 82.2 & &  &  \\
    \enddata
\end{deluxetable*}

\section{Analysis} \label{sec:analysis}

For each epoch we applied a flat correction using calibration frames captured during observing. We also removed bad pixels using a combination of L.A.Cosmic \citep{dokkum_cosmic-ray_2001} and an adaptive sigma-clipping algorithm. We removed sky background using a high-pass median filter with a box size of 31 pixels. At this point frames were manually filtered to remove bad frames, especially those with diffraction spikes within a few hundred pixels, like in \autoref{fig:spike}. Then, each good frame is co-registered using cross-correlations \citep{guizar-sicairos_efficient_2008} as a first-order correction and then fitting each frame with a Gaussian PSF. The co-registered frames are then shifted to the center of the FoV. Lastly the frames were cropped to the inner 200 pixels and stacked into data cubes for each epoch. With the pixel scale of \qty{10}{mas\per px} the crop corresponds to a maximum separation of \qty{1}{"} or a projected separation of \qty{2.7}{\au}. We also measure the parallactic angle of each frame, including corrections for distortion effects following \cite{yelda_improving_2010}. For each epoch, we measure the full-width at half-maximum (FWHM) of the stellar PSF by fitting a bivariate Gaussian model to the median frame from each data cube (an example of one is shown in \autoref{fig:psf}). All of the pre-processing code is available in Jupyter notebooks at the following GitHub repository\footnote{\href{https://github.com/mileslucas/sirius-b}{https://github.com/mileslucas/sirius-b}} and the reduced data cubes and parallactic angles have been made available on Zenodo \citep{lucas_nirc2_2021}.

\begin{figure}
    \centering
    \epsscale{1}
    \plotone{psf}
    \caption{The median frame from the 2020-11-21 epoch showing the instrumental PSF. The inner core has a FWHM of $\sim$\qty{76}{mas}. The speckle pattern can be seen in the blobs surrounding the first ring, with roughly 6-way radial symmetry corresponding to the hexagonal shapes of the segmented mirrors.}
    \label{fig:psf}
\end{figure}

By taking data with the field rotator disabled (ADI), the point-spread function (PSF), which is an instrumental effect, will not appear to rotate while any potential companion will appear to rotate. This allows modeling and subtracting the PSF with less probability for subtracting companion signal. After subtraction, the frames are derotated by their parallactic angle and combined with a weighted sum \citep{bottom_noise-weighted_2017}, which reduces the pixel-to-pixel noise as the number of frames in the data cube increases.

For this analysis we used four statistical models for modeling the stellar PSF, classic median subtraction \citep{marois_angular_2006}, principal component analysis \citep[PCA;][]{soummer_detection_2012}, non-negative matrix factorization \citep[NMF;][]{ren_non-negative_2018}, and fixed-point greedy disk subtraction \citep[GreeDS;][]{pairet_reference-less_2019,pairet_mayonnaise_2020}. The median subtraction and PCA methods were also applied in an annular method, where we model the PSF in annuli of increasing separation frame-by-frame, discarding frames which have not rotated at least 0.5 FWHM \citep{marois_angular_2006}.

We used three metrics for determining the performance of each algorithm, the signal-to-noise ratio (S/N) significance map, the standardized trajectory intensity mean map \citep[STIM map;][]{pairet_stim_2019}, and the contrast curve. The significance and STIM maps assign a likelihood to each pixel for the presence of a companion using different assumptions of the residual statistics. The contrast curve determines the sensitivity of a 5$\sigma$ statistical detection through repeated injection and retrieval of planetary signal as processed by one of the ADI algorithms above. We calculate both the Gaussian contrast and the Student-t corrected contrast, which accounts for the small-sample statistics in each annulus \citep{mawet_fundamental_2014}. The collapsed residual frames along with the above metrics for each algorithm for each epoch can be found in \autoref{sec:adi-results}.

Another metric used was the STIM largest intensity mask map \citep[SLIM map;][]{pairet_signal_2020}. The SLIM map is an ensemble statistic which calculates the average STIM map from many residual cubes along with the average mask of the $N$ most intense pixels in each STIM map. We use the SLIM map with the PCA, NMF, and GreeDS algorithms because they all depend on choosing the number of basis components, which is hard to determine \textit{a priori}. For each algorithm, we create residual cubes for increasing number of components, from 1 to 10. The collapsed residual frames, average STIM map, SLIM map, and contrast curves for each epoch for each of the above algorithms can be found in \autoref{sec:adi-results}. All of the code for these reductions can be found at the GitHub repository\footnote{\href{https://github.com/mileslucas/sirius-b}{https://github.com/mileslucas/sirius-b}}.

\section{Results} \label{sec:results}

\begin{figure*}
    \centering
    \epsscale{1}
    \plotone{residuals}
    \caption{The flat residuals of each epoch after PSF subtraction, derotating, and collapsing. The inner full-width at half-maximum (FWHM) is masked out for each frame.}
    \label{fig:residuals}
\end{figure*}

\begin{figure*}
    \centering
    \epsscale{1}
    \plotone{sig}
    \caption{The \textit{significance} maps for each epoch accounting for small sample statistics \citep{mawet_fundamental_2014}. Typically a critical value for detection is 5. The inner full-width at half-maximum (FWHM) is masked out for each map.}
    \label{fig:sig}
\end{figure*}

\begin{figure*}
    \centering
    \epsscale{1}
    \plotone{stim}
    \caption{The STIM maps for each epoch calculated from the residual cube. Note that the STIM probability has a typical cutoff threshold of 0.5 for significant detections. The inner full-width at half-maximum (FWHM) is masked out for each map.}
    \label{fig:stim}
\end{figure*}

\begin{figure*}[t]
    \centering
    \epsscale{1}
    \plotone{contrast_curves}
    \caption{The contrast curves for the best performing algorithm from each epoch. The solid lines are the Gaussian 5$\sigma$ contrast curves and the dashed lines are the Student-t corrected cuves \citep{mawet_fundamental_2014}. In addition, the expected upper limit for orbital separation of a stable orbit \citep{bond_sirius_2017} of \qty{1.5}{\au} are plotted in a vertical dashed line. The companion mass values are interpolated from the AMES-Cond grid \citep{allard_models_2012}. The lower mass limit (upper magnitude limit) of these models is plotted in a light-gray horizontal dashed line.}
    \label{fig:contrast}
\end{figure*}

We determined the best-performing algorithms for each epoch using the contrast curves described in \autoref{sec:analysis}. For the 2020-02-04 and 2020-11-21 epochs full-frame median subtraction had the best contrast at all separations. For the 2020-11-28 epoch annular PCA subtraction with 2 principal components and a rotation threshold of 0.5 FWHM produced the best contrast at close separations (\qtyrange{0.5}{1}{"}) and had similar performance to other algorithms beyond \qty{1}{"}. The collapsed residual frames from each epoch are shown in \autoref{fig:residuals}, along with the Gaussian significance maps (\autoref{fig:sig}) and STIM maps (\autoref{fig:stim}).

In these images there is not \textit{consistent} or overwhelming evidence for a substellar companion. The STIM probability maps for the 2020-11-21 and 2020-11-28 epochs suggest evidence for some blobs $\sim$\qty{0.13}{"} from the center. The lack of evidence in the February epoch and the significance maps as well as the proximity to the central star ($\sim$2 FWHM) all discredit the probability of these blobs being companions. Nonetheless, we estimated astrometry for blobs from each epoch (\autoref{tbl:astrometry}) and tried fitting Keplerian orbits using the ``Orbits for the Impatient" algorithm \citep[OFTI;][]{blunt_orbits_2017}. We generated $10^4$ orbits, none of which managed to contain the points from each epoch (\autoref{sec:orbits}). We take this as direct evidence against the blobs being substellar companions of any kind.

The contrast maps from the best performing algorithm for each reduction are shown in \autoref{fig:contrast}. We determine the limiting sensitivities in terms of the planetary mass by first calculating the contrast-limited magnitude using an Lp-band magnitude for Sirius B of 9.1 (adapted from \citealp{bonnet-bidaud_adonis_2008}). Then we use an age of \qty{226}{\mega\year} to interpolate the planetary mass using the AMES-Cond evolutionary grid and atmosphere models \citep{allard_models_2012}. It is important to note how the precision of the system age (see \autoref{sec:intro}) leads to quite accurate interpolation on the evolutionary grids. The best performing epoch was on the night of 2020-11-21, which managed to reach an exceptional sensitivity of \qtylist[list-units=single]{3.5;1.9;1.1;0.72;0.63}{\mj} at projected separations of \qtylist[list-units=single]{0.25;0.5;0.75;1;2}{\au} (\qtylist[list-units=single]{0.09;0.19;0.28;0.38;0.75}{\arcsecond}) respectively.


\section{Conclusions} \label{sec:conclusion}

In this work we demonstrate the results of high-contrast images of Sirius B in the near-IR. Our sensitivy limits are the best that have been reached for Sirius B, reaching \qty{3.5}{\mj} at \qty{0.25}{\au}. In the mid-IR previous works have only probed as close-in as $\sim$\qty{1}{\au}, where we reach a sensitivity of \qty{0.72}{\mj}. Despite the exceptional sensitivity of this study, we found no appreciable evidence for a companion around Sirius B, consistent with previous results.

We have published alongside this work the entire codebase used for pre-processing the data, doing the ADI reductions, and for generating every figure in this manuscript. We have also published our reduced datasets under an open license. We hope that this improves the reproducibility of the work as well as providing data for exploring new and different ADI algorithms.

This study also shows the capability for near-IR non-coronagraphic imaging for faint white dwarfs. Our sensitivity limits are deeper and closer than any other WD imaging results, even those using coronagraphs, thanks in no small part to the proximity of Sirius B.

It is interesting to note the morphology of the innermost $\sim$\ang{;;0.4} in the frames produced by GreeDS and NMF. Both of these algorithms have been shown to outperform traditional median and PCA subtraction for disc imaging. In the frames from each epoch, but particularly in the two November epochs, a symmetric ``barbell'' shape can be seen. Due to the nature of high-contrast imaging, it is extremely difficult to differentiate systematic noise from real signal in the speckle-limited regime, in addition there is no evidence for IR excess from a circumstellar disc. Follow-up work in the visible (e.g., Subaru/VAMPIRES, VLT/SPHERE) may be able to image such a disc.

As this study and others have probed deeper and closer to Sirius B, we must consider, too, the astrophysics of post-main-sequence evolution. As mentioned in \autoref{sec:intro}, the expansion of the star during the RGB phase places a limit on the innermost orbital separation that would escape engulfment or strong tidal forces. A study into the evolution of the Sirius B progenitor through the RGB phase could provide these limits; it is possible we have reached those limits observationally and could provide some definitive answer to whether Sirius B has a companion.

\begin{acknowledgments}

\end{acknowledgments}

\software{
ADI.jl \citep{lucas_adijl_2020},
astropy \citep{collaboration_astropy_2013,astropy_collaboration_astropy_2018},
Julia \citep{bezanson_julia_2017},
numpy \citep{harris_array_2020},
scikit-image \citep{walt_scikit-image_2014},
}

\bibliography{references}{}
\bibliographystyle{aasjournal}

\appendix

\section{ADI Processing Results} \label{sec:adi-results}


\figsetstart
\figsetnum{6}
\figsettitle{ADI processing results}

\figsetgrpstart
\figsetgrpnum{6.1}
\figsetgrptitle{2020-02-04 median}
\figsetplot{reports/2020feb04_median}
\figsetgrpnote{Results from processing each epoch of Sirius B data with various ADI algorithms. The top-left image is the subtracted, derotated, and collapsed image. The top-middle image is the Gaussian S/N map, the bottom-middle image is the Student-t S/N (significance) map, which typically requires a value of 5 to show a significant detection, and the bottom-left image is the STIM probability map, which typically requires a probability of 0.5 for significant detection. The right graph shows the 5$\sigma$ contrast, both the Gaussian and Student-t corrected curves.}
\figsetgrpend

\figsetgrpstart
\figsetgrpnum{6.2}
\figsetgrptitle{2020-02-04 PCA(2)}
\figsetplot{reports/2020feb04_pca-2}
\figsetgrpnote{Results from processing each epoch of Sirius B data with various ADI algorithms. The top-left image is the subtracted, derotated, and collapsed image. The top-middle image is the Gaussian S/N map, the bottom-middle image is the Student-t S/N (significance) map, which typically requires a value of 5 to show a significant detection, and the bottom-left image is the STIM probability map, which typically requires a probability of 0.5 for significant detection. The right graph shows the 5$\sigma$ contrast, both the Gaussian and Student-t corrected curves.}
\figsetgrpend

\figsetgrpstart
\figsetgrpnum{6.3}
\figsetgrptitle{2020-02-04 NMF(2))}
\figsetplot{reports/2020feb04_nmf-2}
\figsetgrpnote{Results from processing each epoch of Sirius B data with various ADI algorithms. The top-left image is the subtracted, derotated, and collapsed image. The top-middle image is the Gaussian S/N map, the bottom-middle image is the Student-t S/N (significance) map, which typically requires a value of 5 to show a significant detection, and the bottom-left image is the STIM probability map, which typically requires a probability of 0.5 for significant detection. The right graph shows the 5$\sigma$ contrast, both the Gaussian and Student-t corrected curves.}
\figsetgrpend

\figsetgrpstart
\figsetgrpnum{6.4}
\figsetgrptitle{2020-02-04 GreeDS(2)}
\figsetplot{reports/2020feb04_greeds-2}
\figsetgrpnote{Results from processing each epoch of Sirius B data with various ADI algorithms. The top-left image is the subtracted, derotated, and collapsed image. The top-middle image is the Gaussian S/N map, the bottom-middle image is the Student-t S/N (significance) map, which typically requires a value of 5 to show a significant detection, and the bottom-left image is the STIM probability map, which typically requires a probability of 0.5 for significant detection. The right graph shows the 5$\sigma$ contrast, both the Gaussian and Student-t corrected curves.}
\figsetgrpend

\figsetgrpstart
\figsetgrpnum{6.5}
\figsetgrptitle{2020-02-04 annular median}
\figsetplot{reports/2020feb04_annular_median}
\figsetgrpnote{Results from processing each epoch of Sirius B data with various ADI algorithms. The top-left image is the subtracted, derotated, and collapsed image. The top-middle image is the Gaussian S/N map, the bottom-middle image is the Student-t S/N (significance) map, which typically requires a value of 5 to show a significant detection, and the bottom-left image is the STIM probability map, which typically requires a probability of 0.5 for significant detection. The right graph shows the 5$\sigma$ contrast, both the Gaussian and Student-t corrected curves.}
\figsetgrpend

\figsetgrpstart
\figsetgrpnum{6.6}
\figsetgrptitle{2020-02-04 annular PCA(2))}
\figsetplot{reports/2020feb04_annular_pca-2}
\figsetgrpnote{Results from processing each epoch of Sirius B data with various ADI algorithms. The top-left image is the subtracted, derotated, and collapsed image. The top-middle image is the Gaussian S/N map, the bottom-middle image is the Student-t S/N (significance) map, which typically requires a value of 5 to show a significant detection, and the bottom-left image is the STIM probability map, which typically requires a probability of 0.5 for significant detection. The right graph shows the 5$\sigma$ contrast, both the Gaussian and Student-t corrected curves.}
\figsetgrpend

\figsetgrpstart
\figsetgrpnum{6.7}
\figsetgrptitle{2020-02-04 annular NMF(2)}
\figsetplot{reports/2020feb04_annular_nmf-2}
\figsetgrpnote{Results from processing each epoch of Sirius B data with various ADI algorithms. The top-left image is the subtracted, derotated, and collapsed image. The top-middle image is the Gaussian S/N map, the bottom-middle image is the Student-t S/N (significance) map, which typically requires a value of 5 to show a significant detection, and the bottom-left image is the STIM probability map, which typically requires a probability of 0.5 for significant detection. The right graph shows the 5$\sigma$ contrast, both the Gaussian and Student-t corrected curves.}
\figsetgrpend

\figsetgrpstart
\figsetgrpnum{6.8}
\figsetgrptitle{2020-11-21 median subtraction.}
\figsetplot{reports/2020nov21_median}
\figsetgrpnote{Results from processing each epoch of Sirius B data with various ADI algorithms. The top-left image is the subtracted, derotated, and collapsed image. The top-middle image is the Gaussian S/N map, the bottom-middle image is the Student-t S/N (significance) map, which typically requires a value of 5 to show a significant detection, and the bottom-left image is the STIM probability map, which typically requires a probability of 0.5 for significant detection. The right graph shows the 5$\sigma$ contrast, both the Gaussian and Student-t corrected curves.}
\figsetgrpend

\figsetgrpstart
\figsetgrpnum{6.9}
\figsetgrptitle{2020-11-21 PCA subtraction with 2 components.}
\figsetplot{reports/2020nov21_pca-2}
\figsetgrpnote{Results from processing each epoch of Sirius B data with various ADI algorithms. The top-left image is the subtracted, derotated, and collapsed image. The top-middle image is the Gaussian S/N map, the bottom-middle image is the Student-t S/N (significance) map, which typically requires a value of 5 to show a significant detection, and the bottom-left image is the STIM probability map, which typically requires a probability of 0.5 for significant detection. The right graph shows the 5$\sigma$ contrast, both the Gaussian and Student-t corrected curves.}
\figsetgrpend

\figsetgrpstart
\figsetgrpnum{6.10}
\figsetgrptitle{2020-11-21 NMF subtraction with 2 components.}
\figsetplot{reports/2020nov21_nmf-2}
\figsetgrpnote{Results from processing each epoch of Sirius B data with various ADI algorithms. The top-left image is the subtracted, derotated, and collapsed image. The top-middle image is the Gaussian S/N map, the bottom-middle image is the Student-t S/N (significance) map, which typically requires a value of 5 to show a significant detection, and the bottom-left image is the STIM probability map, which typically requires a probability of 0.5 for significant detection. The right graph shows the 5$\sigma$ contrast, both the Gaussian and Student-t corrected curves.}
\figsetgrpend

\figsetgrpstart
\figsetgrpnum{6.11}
\figsetgrptitle{2020-11-21 GreeDS with 2 components.}
\figsetplot{reports/2020nov21_greeds-2}
\figsetgrpnote{Results from processing each epoch of Sirius B data with various ADI algorithms. The top-left image is the subtracted, derotated, and collapsed image. The top-middle image is the Gaussian S/N map, the bottom-middle image is the Student-t S/N (significance) map, which typically requires a value of 5 to show a significant detection, and the bottom-left image is the STIM probability map, which typically requires a probability of 0.5 for significant detection. The right graph shows the 5$\sigma$ contrast, both the Gaussian and Student-t corrected curves.}
\figsetgrpend

\figsetgrpstart
\figsetgrpnum{6.12}
\figsetgrptitle{2020-11-21 annular median subtraction with a rotation threshold of 0.5.}
\figsetplot{reports/2020nov21_annular_median}
\figsetgrpnote{Results from processing each epoch of Sirius B data with various ADI algorithms. The top-left image is the subtracted, derotated, and collapsed image. The top-middle image is the Gaussian S/N map, the bottom-middle image is the Student-t S/N (significance) map, which typically requires a value of 5 to show a significant detection, and the bottom-left image is the STIM probability map, which typically requires a probability of 0.5 for significant detection. The right graph shows the 5$\sigma$ contrast, both the Gaussian and Student-t corrected curves.}
\figsetgrpend

\figsetgrpstart
\figsetgrpnum{6.13}
\figsetgrptitle{2020-11-21 annular PCA subtraction with 2 components and a rotation threshold ofnov21 0.5.}
\figsetplot{reports/2020nov21_annular_pca-2}
\figsetgrpnote{Results from processing each epoch of Sirius B data with various ADI algorithms. The top-left image is the subtracted, derotated, and collapsed image. The top-middle image is the Gaussian S/N map, the bottom-middle image is the Student-t S/N (significance) map, which typically requires a value of 5 to show a significant detection, and the bottom-left image is the STIM probability map, which typically requires a probability of 0.5 for significant detection. The right graph shows the 5$\sigma$ contrast, both the Gaussian and Student-t corrected curves.}
\figsetgrpend

\figsetgrpstart
\figsetgrpnum{6.14}
\figsetgrptitle{2020-11-21 annular NMF subtraction with 2 components and a rotation threshold of 0.5.}
\figsetplot{reports/2020nov21_annular_nmf-2}
\figsetgrpnote{Results from processing each epoch of Sirius B data with various ADI algorithms. The top-left image is the subtracted, derotated, and collapsed image. The top-middle image is the Gaussian S/N map, the bottom-middle image is the Student-t S/N (significance) map, which typically requires a value of 5 to show a significant detection, and the bottom-left image is the STIM probability map, which typically requires a probability of 0.5 for significant detection. The right graph shows the 5$\sigma$ contrast, both the Gaussian and Student-t corrected curves.}
\figsetgrpend

\figsetgrpstart
\figsetgrpnum{6.15}
\figsetgrptitle{2020-11-28 median subtraction.}
\figsetplot{reports/2020nov28_median}
\figsetgrpnote{Results from processing each epoch of Sirius B data with various ADI algorithms. The top-left image is the subtracted, derotated, and collapsed image. The top-middle image is the Gaussian S/N map, the bottom-middle image is the Student-t S/N (significance) map, which typically requires a value of 5 to show a significant detection, and the bottom-left image is the STIM probability map, which typically requires a probability of 0.5 for significant detection. The right graph shows the 5$\sigma$ contrast, both the Gaussian and Student-t corrected curves.}
\figsetgrpend

\figsetgrpstart
\figsetgrpnum{6.16}
\figsetgrptitle{2020-11-28 PCA subtraction with 2 components.}
\figsetplot{reports/2020nov28_pca-2}
\figsetgrpnote{Results from processing each epoch of Sirius B data with various ADI algorithms. The top-left image is the subtracted, derotated, and collapsed image. The top-middle image is the Gaussian S/N map, the bottom-middle image is the Student-t S/N (significance) map, which typically requires a value of 5 to show a significant detection, and the bottom-left image is the STIM probability map, which typically requires a probability of 0.5 for significant detection. The right graph shows the 5$\sigma$ contrast, both the Gaussian and Student-t corrected curves.}
\figsetgrpend

\figsetgrpstart
\figsetgrpnum{6.17}
\figsetgrptitle{2020-11-28 NMF subtraction with 2 components.}
\figsetplot{reports/2020nov28_nmf-2}
\figsetgrpnote{Results from processing each epoch of Sirius B data with various ADI algorithms. The top-left image is the subtracted, derotated, and collapsed image. The top-middle image is the Gaussian S/N map, the bottom-middle image is the Student-t S/N (significance) map, which typically requires a value of 5 to show a significant detection, and the bottom-left image is the STIM probability map, which typically requires a probability of 0.5 for significant detection. The right graph shows the 5$\sigma$ contrast, both the Gaussian and Student-t corrected curves.}
\figsetgrpend

\figsetgrpstart
\figsetgrpnum{6.18}
\figsetgrptitle{2020-11-28 GreeDS with 2 components.}
\figsetplot{reports/2020nov28_greeds-2}
\figsetgrpnote{Results from processing each epoch of Sirius B data with various ADI algorithms. The top-left image is the subtracted, derotated, and collapsed image. The top-middle image is the Gaussian S/N map, the bottom-middle image is the Student-t S/N (significance) map, which typically requires a value of 5 to show a significant detection, and the bottom-left image is the STIM probability map, which typically requires a probability of 0.5 for significant detection. The right graph shows the 5$\sigma$ contrast, both the Gaussian and Student-t corrected curves.}
\figsetgrpend

\figsetgrpstart
\figsetgrpnum{6.19}
\figsetgrptitle{2020-11-28 annular median subtraction with a rotation threshold of 0.5.}
\figsetplot{reports/2020nov28_annular_median}
\figsetgrpnote{Results from processing each epoch of Sirius B data with various ADI algorithms. The top-left image is the subtracted, derotated, and collapsed image. The top-middle image is the Gaussian S/N map, the bottom-middle image is the Student-t S/N (significance) map, which typically requires a value of 5 to show a significant detection, and the bottom-left image is the STIM probability map, which typically requires a probability of 0.5 for significant detection. The right graph shows the 5$\sigma$ contrast, both the Gaussian and Student-t corrected curves.}
\figsetgrpend

\figsetgrpstart
\figsetgrpnum{6.20}
\figsetgrptitle{2020-11-28 annular PCA subtraction with 2 components and a rotation threshold ofnov28 0.5.}
\figsetplot{reports/2020nov28_annular_pca-2}
\figsetgrpnote{Results from processing each epoch of Sirius B data with various ADI algorithms. The top-left image is the subtracted, derotated, and collapsed image. The top-middle image is the Gaussian S/N map, the bottom-middle image is the Student-t S/N (significance) map, which typically requires a value of 5 to show a significant detection, and the bottom-left image is the STIM probability map, which typically requires a probability of 0.5 for significant detection. The right graph shows the 5$\sigma$ contrast, both the Gaussian and Student-t corrected curves.}
\figsetgrpend

\figsetgrpstart
\figsetgrpnum{6.21}
\figsetgrptitle{2020-11-28 annular NMF subtraction with 2 components and a rotation threshold of 0.5.}
\figsetplot{reports/2020nov28_annular_nmf-2}
\figsetgrpnote{Results from processing each epoch of Sirius B data with various ADI algorithms. The top-left image is the subtracted, derotated, and collapsed image. The top-middle image is the Gaussian S/N map, the bottom-middle image is the Student-t S/N (significance) map, which typically requires a value of 5 to show a significant detection, and the bottom-left image is the STIM probability map, which typically requires a probability of 0.5 for significant detection. The right graph shows the 5$\sigma$ contrast, both the Gaussian and Student-t corrected curves.}
\figsetgrpend

\figsetend

\begin{figure*}
\figurenum{6}
\plotone{reports/2020feb04_median}
\caption{Results from processing each epoch of Sirius B data with various ADI algorithms. The top-left image is the subtracted, derotated, and collapsed image. The top-middle image is the Gaussian S/N map, the bottom-middle image is the Student-t S/N (significance) map, which typically requires a value of 5 to show a significant detection, and the bottom-left image is the STIM probability map, which typically requires a probability of 0.5 for significant detection. The right graph shows the 5$\sigma$ contrast, both the Gaussian and Student-t corrected curves.}
\end{figure*}



\begin{figure*}
    \plotone{2020feb04_contrast_curves}
    \caption{5$\sigma$ contrast curves from various ADI algorithms for the 2020-02-04 epoch. Both the Gaussian (solid lines) and Student-t corrected (dashed lines) contrast curves are shown.}
\end{figure*}

\begin{figure*}
    \plotone{2020nov21_contrast_curves}
    \caption{5$\sigma$ contrast curves from various ADI algorithms for the 2020-11-21 epoch. Both the Gaussian (solid lines) and Student-t corrected (dashed lines) contrast curves are shown.}
\end{figure*}

\begin{figure*}
    \plotone{2020nov28_contrast_curves}
    \caption{5$\sigma$ contrast curves from various ADI algorithms for the 2020-11-28 epoch. Both the Gaussian (solid lines) and Student-t corrected (dashed lines) contrast curves are shown.}
\end{figure*}

\section{Provisional Orbit Fitting} \label{sec:orbits}


\begin{deluxetable*}{ccc}
    \tablecaption{Provisional astrometry for blobs of interesest from each epoch. The uncertainties are represented in parantheses and are derived from the FWHM of the PSF from each epoch.
    \label{tbl:astrometry}}
    \tablehead{
        \colhead{Date observed} &
        \colhead{offset (mas)} &
        \colhead{PA (\textdegree)}
    }
    \startdata
    2020-02-04 & 193(38) & -99(11)  \\
    2020-11-21 & 125(38) & 75(17)  \\
    2020-11-28 & 215(40) & 23(11)  \\
    \enddata
\end{deluxetable*}

\begin{figure*}
    \plotone{orbit_frames}
    \caption{Provisional astrometry (white circles) displayed on STIM maps using the GreeDS algorithm with 2 components. Each frame was been cropped to the inner \ang{;;0.25}. The inner FWHM has been masked out in each frame.}
\end{figure*}

\begin{figure*}
    \plotone{orbit_corner}
    \caption{Corner plot of the marginal posteriors from the OFTI samples.}
\end{figure*}

\end{document}
